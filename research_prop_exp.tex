% !TEX root = ./diss_proposal.tex

\section{Proposed analyses}

\subsection{Similarity}

In relation to Goal 3 (presented in the Proposed dependent measures section): Perform similarity analyses between test events and the pairs of study events.  Deficient processing would predict that spaced stimuli should have a good match between test, the initial presentation, and the repetition; however, for massed stimuli the test should only match the initial presentation.  This would be the case regardless of whether the principal components reflect memory contents or active neural mechanisms.

For contextual variability, if components reflect memory contents, spaced test presentations should match either the initial presentation or the repetition, but not both.  Both would be matched for massed items, as well as if the components reflect active processes.
Study-phase retrieval would predict that a test stimulus would match both study presentations.

\subsection{Classification}

The classification analyses described earlier in Goal 4 need to be performed.

\section{Proposed experiments}

Two experiments are proposed, but running only one should be sufficient to answer further questions about the spacing effect.  We expect that both will replicate the effects seen so far and that either would further our understanding of the spacing effect.

\subsection{Experiment~2a: Additional Study Lags}

This experiment involves a design that should replicate Experiment~1 while extending it to include other lags.  The purpose of this is to look for gradations in EEG effects so we can better interpret data patterns that fit multiple theories.  Experiment~1 used lags of 0 and 12; this experiment could keep these lags and add repetitions at lags of 4 and 32 (arbitrarily chosen but within the range of behavioral spacing studies).  Using a recognition test, it would be ideal to make the task more difficult to increase trial counts for missed items.  Two simple ways to do this is by increasing study list length or decreasing study presentation time.

Including the single-presentation stimuli at test would allow us to get a baseline measurement of memory performance for comparison of subsequently remembered and forgotten massed and spaced items.  For example, we would always expect a repetition effect (repeated items will be remembered better than single presentation items), but perhaps if massed repetitions do involve deficient processing then they will be recalled no better than single presentation items.

For analyses, it might be beneficial to present stimuli simultaneously to avoid the complication of having two successive stimuli per trial.  This would eliminate the ability to attempt to detect image-related activity during a word repetition, but the added benefit of simplification may outweigh this loss.

For predictions about effects, we expect that memory performance will correlate positively with lag, known as a lag effect in this literature.  For EEG, our most informative effects regarding the spacing effect were for the N1, upper alpha, and time--frequency similarity, though the latter are difficult to interpret.  Overall, they implicate differences in attention and semantic processing between spaced and massed repetitions that led to interactions with subsequent memory.

With more inter-study lags, we would expect deficient processing to show a graded effect, specifically decreasing with lag.  The N1 should become more negative as lag increases because early attentional processes vary with repetition lag; a repetition at lag 2 would have a voltage between a massed item and a repetition at lag 12.  We would expect semantic processes (N400 and upper alpha) as well as working memory effects that vary with repetition lag (LPC) to show similar gradients: N400 should get more negative as lag increases and upper alpha for remembered spaced items should desynchronize more as lag increases.

\cbstart
An important question to consider regarding the N1 is how short term the repetition effects are.  According to \citeA{HensEtal2004}, who investigated ERP effects for repetitions of pictures of objects at different lags, they saw an N1 repetition effect at after an unfilled 4-second delay, but not when the four seconds was occupied by another stimulus or at a much longer lag (96 seconds).  Thus, deficient processing may be eliminated at a relatively short delay, which is a good reason for including a relatively short lag, such as 4 intervening items.
\cbend

None of our effects have supported contextual variability, but if it is at play we should see more variability in EEG at even longer lags.  The closest effect this is related to in the present results is in time--frequency similarity; perhaps a clearer picture will emerge with more lags.

At longer lags, study-phase retrieval should be more difficult but also more beneficial to long-term memory.  We would expect subsequently remembered long-lag stimuli to show a stronger reinstatement during repetitions.  This would manifest in the theta band, and perhaps would also show stronger recollection memory ERP effects (such as the parietal old/new effect).
It would be enlightening to get away from a paradigm that involves immediate repetition so that deficient processing could be eliminated as a possible theory.  Perhaps a lag effect between spacings of 7 and 36 items would be spaced apart far enough to eliminate deficient processing.  More interesting would be a paradigm with external validity, something that has more of a real-world learning timescale (see Experiment~2b), but this may be too far removed from the current paradigm to make accurate predictions.

\subsection{Experiment~2b: Optimal Spacing Replication}

This is a three-session experiment that involves a longer study gap and retention interval, thereby allowing for analysis of the spacing effect at a timescale that approaches educational relevance.  It is essentially a replication of \citeA[Experiment~1]{CepeEtal2009} using EEG.  The long timescale makes this experiment markedly different from Experiment~1, and its design has both benefits (eliminates deficient processing, an ``impostor'' effect; \citeNP{DelaEtal2010}; spacing effects may be stronger) and drawbacks (particularly that its different design will make direct comparisons to Experiment~1 difficult, and Experiment~1 results were most consistent with deficient processing, so those effects would be unlikely to replicate).

It has the added benefit that the optimal spacing (1 day) for its retention interval (10 days) has been determined both behaviorally \cite{CepeEtal2009} and using a mathematical model \cite{MozeEtal2009}.  We would record EEG during all three sessions, but each session is relatively short compared to Experiment~1.  Whether we use the same word and image stimuli from Experiment~1 or the English--Swahili translations from \citeA[Experiment~1]{CepeEtal2009} remains to be determined; we have a set of approximately 500 English--Swahili terms already translated from a prior experiment.

On session 1, participants first study a set of 40 paired associates, and are then tested (with feedback) using word recall until each pair is learned to criterion (two recalls).  As the optimal spacing gap is one day, they return the next day for session 2 and repeated the test procedure (with feedback) but this time the stimulus set is simply tested twice (pairs are not learned to criterion again).  Finally, session 3 occurs 10 days after the prior session and participants are tested without feedback.
As a side note, our lab has experience with running multi-session EEG experiments.

An important point about doing an exact replication is that there is no ``massed'' condition.  However, there are two options to include this condition.  The retention interval remains at 10 days following session 2 for both options.  (1) We could use a between-subjects design and have one group participate in session 2 immediately following session 1.  (2) We could use a within-subjects design by splitting session 2 into two phases \cite<similar to the design of>{WagnEtal2000}.  All of the stimulus pairs from session 1 (the ``spaced'' pairs) would make up half of the pairs studied during session 2 phase 1.  The other half would be new pairs (the ``massed'' pairs); these are studied again in session 2 phase 2, which immediately follows phase 1.  Importantly, in phase 1 the ``massed'' pairs would need to be learned to criterion while the spaced pairs would only be studied twice.  It still seems unlikely that deficient processing theory would remain relevant, even in option 2.

According to the results of \citeA[Experiment~1]{CepeEtal2009}, recall at the final test was approximately 65\%.  Since there were only 40 pairs in that experiment, we would need to increase the number of learned items to have enough trials for EEG analysis.
%This may balance out recalled and forgotten trials even more.

This design might allow us to examine the match between a neural network model \cite<MCM;>{MozeEtal2009} that makes accurate predictions for spacing effects and our neural data.  This could include estimations of contextual variability (and contextual drift), study-phase retrieval, etc.

A point that may either be positive or negative is that deficient processing should be completely eliminated with a spacing gap of one day.  Deficient processing is not an effect one would expect to encounter in the real world; instead it seems to be a byproduct of processes acting across relatively short timescales, something our Experiment~1 is guilty of \cite{CepeEtal2006}.

% For predictions, we expect to see effects related to encoding and retrieval, and perhaps even encoding variability.
Increasingly large distributed practice effects are seen at longer delays \cite{CepeEtal2009,DelaEtal2010}.  Because of the one-day spacing delay, we would expect to see stronger effects related to study-phase retrieval and/or contextual variability than were seen in Experiment~1.  EEG activity related to retrieval (theta and parietal old/new ERP effects) might be stronger compared to a massed condition that has a lag on the order of minutes.  Additionally, similarity effects might be stronger across a longer timescale; it seems possible that the temporal context that accompanied study repetitions Experiment~1---even at a spacing lag of 12---had not changed much, but context would be highly different 24 hours later.  We cannot think of any reason to expect large effects related to attention or semantic processing; past studies that have used long delays do not discuss these aspects \cite{CepeEtal2009}.

\subsection{Proposed experiment summary}

Running Experiment~1 and Experiment~2a will follow a replicate-and-extend model and likely can answer some remaining questions about the spacing effect by examining patterns of effects across a range of lags.  Since the deficient processing theory is something we would like to step away from (due to its status as an ``impostor effect''; \citeNP{DelaEtal2010}), it seems possible that having multiple lags that last at least a few seconds (e.g., a lag of 4, if not more) will eliminate at least some deficient processing effects, or at least we will see them decrease in a graded fashion.


If Experiment~1 and Experiment~2b are run, this will provide drastically different views of the spacing effect, the contrast of which may be informative especially when deficient processing is completely eliminated.  Analyzing the spacing effect at a timescale that approaches one that might be used in the real world could answer deeper questions about why the spacing effect occurs.

Regardless of which is run, we hope to provide further insight into why this ubiquitous learning effect occurs.

% \subsection{Experiment~2c: Continuous Recognition}

% % testing effect, external validity

% A continuous recognition spacing effect experiment has the benefit of knowing that an item was recognized at the repetition and could therefore reveal when study-phase retrieval occurs.
% However, this paradigm has issues mentioned earlier such as inducing the testing effect \cite[p.~91]{DelaEtal2010} and that it is not particularly relevant to real-world learning.

% \subsection{Possible experiment details to act on}

% \begin{itemize}
% \item Testing the single-presentation stimuli would allow us to get a baseline.  This performance could also be used to make forgetting curve estimates with MCM \cite{MozeEtal2009}, and could help us determine ``optimal'' spacing.  Does it make sense to do a between subjects manipulation with optimal vs random spacing? e.g., could ask whether brain activity differ between these situations.

% \item Having short, medium, and long lags would allow a better test of whether deficient processing is a possible real-world effect.  This would also be a better test of encoding variability.

% \item If doing an associative memory, it would be good to present stimuli simultaneously to avoid the successive word--image analysis complication.

% \item Encoding variability may be easier to investigate/assess with an explicit manipulation of (for example) context or levels-of-processing.  However, I already cited research that showed spacing effects did not usually occur under these conditions.  Of note, there were two experiments that did show spacing effects under these kinds of manipulations (\citeNP[Experiment~3]{BrauRubi1998}; \citeNP{MalmShif2005}).
% \end{itemize}

% \subsection{Experiment~2: Predictions based on a computational model}

% This experiment relies on getting measurements of context representations using MCM \cite{MozeEtal2009}.  \hl{(I emailed briefly with Mike Mozer about this.  It is possible to get mean similarity measurements out of MCM, but he said this might be overkill and just recommended thinking about its qualitative predictions.  It still seems to me like a cool idea if we could make it work and/or get his help.)}  This would allow us to examine the match between a neural network model that makes accurate predictions for spacing effects and our neural data.  This would include estimations of contextual variability (and contextual drift), study-phase retrieval, etc.

% Do an experiment like \citeA{CepeEtal2008} or \citeA{CepeEtal2009} (flashcard-like foreign language learning task) where:
% \begin{itemize}
% \item Session 1: train to criterion.
% \item Session 2: after a given study lag: cued recall, restudy. Would this be a between subjects manipulation? Or can I bring subjects back once for a massed session and once for a spaced session? I don't think separate spaced and massed restudy sessions would work because RI needs to be the same for massed and spaced.
% \item Session 3: after a given retention interval (optimal, as determined by MCM?): cued recall.
% \end{itemize}
% We already have English--Swahili word translations.  Importantly, stimuli would need to be presented successively and could not use simultaneous presentation as mentioned above.

% \section{Previously proposed experiments}

% \hl{These previously proposed experiments have not been modified since last time.}

% \subsection{Experiment 1b}

% \begin{itemize}
%     \item Like Experiment~1, but with multiple lags.
%     \item Repetitions: single-presentation and repeated items.
%     \item Test: Like Experiment 1, but include single-presentation items.
% \end{itemize}

% \subsection{Experiment 1c}

% \begin{itemize}
%     \item Like Experiment~1, but with multiple lags and across multiple days (same retention interval between Study 2 and test).
%     \item Repetitions: single-presentation and repeated items.
%     \item Massed: Study 1 on Day 1, Study 2 on Day 2, and Test on Day 5.
%     \item Spaced: Study 1 on Day 1, Study 2 on Day 5, and Test on Day 8.
%     \item Test: Like Experiment 1, but include single-presentation items as well.
% \end{itemize}

% \subsection{Experiment 2a}

% A continuous recognition would allow us to ensure that we know when repeated items are recognized (study-phase retrieval).

% \begin{itemize}
%     \item Study: View lists of words (basic free recall setup), as well as lists of continuous recognition of words.

%     \item Repetitions: single-presentation and repeated items.
%     \item Spacing: Massed and multiple lags.

%     \item Test: Free recall
% \end{itemize}

% However, it is important to note that continuous recognition could induce a testing effect because each presentation is essentially a test \cite[p.~91]{DelaEtal2010}.

% Additionally, as noted in the background, deliberately varying context might prevent a spacing effect from occurring.

% \subsection{Experiment 2b}

% \begin{itemize}
%     \item Study: Continuous recognition of paired associates. Pairs are presented at the same time. Deliberate contextual variation: words are either presented with house only, face only, or both (mixed: one category for the initial presentation, the other category for the repetition).

%     \item Repetitions: single-presentation and repeated items.
%     \item Spacing: Massed and multiple lags

% % TC: This is a cool manipulation, but not sure what it has to do with the spacing effects, especially since you are predicting separate context effects and spacing effects, rather than interactions.

%     \item Test: Recognition: choice of Context-1, Context-2, or both.  Mixed pairs recognized with Context-1 at test should show more reinstatement of Context-1 during Study Presentation-2 than mixed pairs recognized with both contexts.
%     \item More likely to recognized Mixed Context-2 when missed during continuous recognition study because study-phase retrieval did not occur.
%     \item Better at remembering both contexts for spaced vs massed.
% \end{itemize}
