% !TEX root = ./diss.tex

\section{Experiment 2: Additional Study Lags}

Running Experiment~1 and Experiment~2a (named this way in the proposal, simply called Experiment~2 here) will follow a replicate-and-extend model and likely can answer some remaining questions about the spacing effect by examining patterns of effects across a range of lags.  Since the deficient processing theory is something we would like to step away from (due to its status as an ``impostor effect''; \citeNP{DelaEtal2010}), it seems possible that having multiple lags that last at least a few seconds (e.g., a lag of 4, if not more) will eliminate at least some deficient processing effects, or at least we will see them decrease in a graded fashion.

\subsection{Design and Predictions}

This experiment involves a design that should replicate Experiment~1 while extending it to include other lags.  The purpose of this is to look for gradations in EEG effects so we can better interpret data patterns that fit multiple theories.  Experiment~1 used lags of 0 and 12; this experiment could keep these lags and add repetitions at lags of 4 and 32 (arbitrarily chosen but within the range of behavioral spacing studies).  Using a recognition test, it would be ideal to make the task more difficult to increase trial counts for missed items.  Two simple ways to do this is by increasing study list length or decreasing study presentation time.

Including the single-presentation stimuli at test would allow us to get a baseline measurement of memory performance for comparison of subsequently remembered and forgotten massed and spaced items.  For example, we would always expect a repetition effect (repeated items will be remembered better than single presentation items), but perhaps if massed repetitions do involve deficient processing then they will be recalled no better than single presentation items.

For analyses, it might be beneficial to present stimuli simultaneously to avoid the complication of having two successive stimuli per trial.  This would eliminate the ability to attempt to detect image-related activity during a word repetition, but the added benefit of simplification may outweigh this loss.

For predictions about effects, we expect that memory performance will correlate positively with lag, known as a lag effect in this literature.  For EEG, our most informative effects regarding the spacing effect were for the N1, upper alpha, and time--frequency similarity, though the latter are difficult to interpret.  Overall, they implicate differences in attention and semantic processing between spaced and massed repetitions that led to interactions with subsequent memory.

With more inter-study lags, we would expect deficient processing to show a graded effect, specifically decreasing with lag.  The N1 should become more negative as lag increases because early attentional processes vary with repetition lag; a repetition at lag 2 would have a voltage between a massed item and a repetition at lag 12.  We would expect semantic processes (N400 and upper alpha) as well as working memory effects that vary with repetition lag (LPC) to show similar gradients: N400 should get more negative as lag increases and upper alpha for remembered spaced items should desynchronize more as lag increases.

An important question to consider regarding the N1 is how short term the repetition effects are.  According to \citeA{HensEtal2004}, who investigated ERP effects for repetitions of pictures of objects at different lags, they saw an N1 repetition effect at after an unfilled 4-second delay, but not when the four seconds was occupied by another stimulus or at a much longer lag (96 seconds).  Thus, deficient processing may be eliminated at a relatively short delay, which is a good reason for including a relatively short lag, such as 4 intervening items.

None of our effects have supported contextual variability, but if it is at play we should see more variability in EEG at even longer lags.  The closest effect this is related to in the present results is in time--frequency similarity; perhaps a clearer picture will emerge with more lags.

At longer lags, study-phase retrieval should be more difficult but also more beneficial to long-term memory.  We would expect subsequently remembered long-lag stimuli to show a stronger reinstatement during repetitions.  This would manifest in the theta band, and perhaps would also show stronger recollection memory ERP effects (such as the parietal old/new effect).
It would be enlightening to get away from a paradigm that involves immediate repetition so that deficient processing could be eliminated as a possible theory.  Perhaps a lag effect between spacings of 7 and 36 items would be spaced apart far enough to eliminate deficient processing.  More interesting would be a paradigm with external validity, something that has more of a real-world learning timescale (see Experiment~2b), but this may be too far removed from the current paradigm to make accurate predictions.

