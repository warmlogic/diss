% !TEX root = ./diss.tex

\section{Background}
% \section{Background and Significance}

% Each of the three theories is described in more detail below.  This includes analysis goals to assess the dependent measures that are implicated in each theory.

% a few subsections with relevant information, somewhere between 1 and
% 3 paragraphs each. describe competing hypotheses here.

%%%%%%%%%%%%%%%%%%%%%%%%%%%%%%%%%%%%%%%%%%%%%%%%%%%%%%%%%%%%
% Deficient processing
%%%%%%%%%%%%%%%%%%%%%%%%%%%%%%%%%%%%%%%%%%%%%%%%%%%%%%%%%%%%

\subsection{Deficient Processing}

The deficient processing theory focuses on the cognitive processes that are active during encoding.  This hypothesis predicts that when an item is repeated in a massed fashion (immediate repetition), because it is familiar and already in working memory there is a greater decrease in attention or encoding effort to the second presentation than there would be to a novel item or spaced repetition.  According to \citeA{Gree1989a}, this could manifest as less rehearsal relative to a spaced repetition due to the stimulus's level of familiarity, and not to an experimental variable such as a difference in the relative amount of rehearsal time available for massed and spaced items.  This could happen either voluntarily through the explicit control of attention \cite{Gree1989a}, or involuntarily via a habituation mechanism like neural repetition suppression \cite{CallSchw2010,Hint1974,VanSEtal2007,WagnEtal2000,XueEtal2011} or via short-term priming and transfer-appropriate processing \cite{Chal1993,MammEtal2002,RussEtal1998}.


%%%%%%%%%%%%%%%%%%%%%%%%%%
% DP problem: reliance on working memory
%%%%%%%%%%%%%%%%%%%%%%%%%%

Even though this theory has been supported by recent research examining the neural basis of the spacing effect \cite<specifically in relation to neural repetition suppression;>{CallSchw2010,VanSEtal2007,XueEtal2011}, the deficient processing hypothesis cannot completely explain the spacing effect for a few reasons.  The first is because of its reliance on working memory, such that it only seems to be a viable explanation at short timescales.  Past research has shown that the spacing effect occurs across multiple days, weeks, and even months \cite{CepeEtal2006,CepeEtal2009}.
%; e.g., the second ``massed'' study episode occurs on the following day while the second ``spaced'' study episode occurs a week later \hl{(explain this better, esp in relation to when the test occurs)}.
It might be possible to test whether different extents of deficient processing occur using a gradation of spaced inter-study lags (e.g., deficient processing should decrease as lag increases), but other mechanisms are still needed.

% An issue for any theory that relies on the spacing effect's contingent factors occurring solely during encoding is that if you can eliminate the spacing effect with a retrieval manipulation, then the account cannot provide a complete explanation \cite<deficient processing was mentioned by>{Gree1989a}.  If the spacing effect can be diminished through a manipulation that occurs during the subsequent memory test then the manipulations that occurred during encoding (e.g., level of processing) should not matter.

%%%%%%%%%%%%%%%%%%%%%%%%%%
% DP problem: massed disadvantage
%%%%%%%%%%%%%%%%%%%%%%%%%%

Second, \citeA{DelaEtal2010} called deficient processing an ``impostor'' spacing effect.  Massed items are put at a disadvantage rather than giving the spaced items an advantage, and so this theory accounts for the spacing effect with a relative difference between the two.  Since we want to know about a true spacing effect (i.e., an advantage for spaced items), we must assume that other mechanisms are at play.

%%%%%%%%%%%%%%%%%%%%%%%%%%
% DP problem: no context
%%%%%%%%%%%%%%%%%%%%%%%%%%

Finally, decades of theoretical and mathematical accounts of human memory have stressed the importance of the context that accompanies studied material in influencing memory performance, and the deficient processing theory does not consider this aspect. \hl{(Include references or another sentence about context?)}
In summary, the deficient processing effect is still interesting to investigate due to the recent attention it has been given in functional magnetic resonance imaging (fMRI) publications regarding neural repetition suppression, but other theories are needed because it cannot completely account for the spacing effect.

%%%%%%%%%%%%%%%%%%%%%%%%%%%%%%%%%%%%%%%%%%%%%%%%%%%%%%%%%%%%
% Contextual variability
%%%%%%%%%%%%%%%%%%%%%%%%%%%%%%%%%%%%%%%%%%%%%%%%%%%%%%%%%%%%

\subsection{Contextual Variability}

Contextual variability theory assumes that item study presentations are associated with a slowly drifting background context that also gets encoded in a memory trace \cite{Bowe1972,Este1955a,Melt1970}.
% Upon remembering an episode at a later point, perhaps it is this context that allows us to accurately perform source memory judgments.
The set of features that constitute context is not completely consistent across different theories and models, but most accounts agree to a basic extent.  Context typically includes the incidental background stimuli that are present during encoding (e.g., the experimental backdrop and non-relevant stimuli) as well as the internal state of the learner; it drifts or fluctuates as time passes \cite{Bowe1972,DelaEtal2010,Glen1979,MalmShif2005,Raai2003}.  The contextual state can be impressionable, influenced by recent experiences (e.g., the other items in a list) \cite{HowaKaha2002,SedeEtal2008}. This means that context can change depending on the information that is encoded or retrieved from memory.  This proposal takes the view that the contextual state fluctuates over time and is influenced by the contents of memory, as this is how spacing effects are typically accounted for in mathematical models of memory.
% like Search of Associative Memory \cite<SAM;>{MensRaai1989}, Retrieving Effectively from Memory \cite<REM;>{ShifStey1997}, Temporal Context Model \cite<TCM;>{HowaKaha2002,SedeEtal2008}, and Context Maintenance and Retrieval \cite<CMR;>{PolyEtal2009}.

%%%%%%%%%%%%%%%%%%%%%%%%%%
% CV: temporocontextual drift, study--test match
%%%%%%%%%%%%%%%%%%%%%%%%%%

Because episodic context drifts with time, the context encoded with repeated study events will differ more as inter-study lag increases \cite{Este1955a}.  This was first proposed as a reason for spacing effect by \citeA{Melt1967} and was integrated into more elaborate theories \cite{Bowe1972,Glen1976,Glen1979}.  Subsequent retrieval of an item during test depends, at least partially, on the similarity between the study and test contexts \cite<in line with the encoding specificity principle>{TulvThom1973}.  Consequently, spaced items are recalled better due to the higher probability that the contextual state at test will match that of the spaced repetitions compared to the less variable (or nearly identical) massed repetitions; essentially, there are more retrieval cues for spaced items.  To account for decreased performance at especially long study repetition lags, \citeA{Glen1976} proposed that the test context would not overlap with the first presentation and memory would rely solely on the second study presentation.


%%%%%%%%%%%%%%%%%%%%%%%%%%
% CV: deliberate/extreme context variation doesn't usually work
%%%%%%%%%%%%%%%%%%%%%%%%%%

As a side note, attempts to deliberately vary the context encoded with massed and spaced repetitions (e.g., changing the paired associate, using homonyms, varying the level of processing), have typically not produced spacing effects \cite<e.g.,>{GlanDuar1971,HintEtal1975a}
%\cite[p.~87--90]{DelaEtal2010},
(but see \citeNP[Experiment~3]{BrauRubi1998}; \citeNP{MalmShif2005}).  This is typically accounted for by assuming that when context is too variable the item is no longer encoded as a repetition.  Therefore, attributing spacing effects to temporocontextual drift rather than strictly defined contextual cues seems more promising \cite{Gree1989a}.

% \citeA{PolyEtal2009}, in their CMR model, accounted for failures to find spacing effects in deliberately varied contexts by claiming that the advantage of contextual variability is counteracted by retrieving prior information that is too different from the present.

%%%%%%%%%%%%%%%%%%%%%%%%%%
% CV: neural similarity during study and test
%%%%%%%%%%%%%%%%%%%%%%%%%%

To extend contextual variability's predictions in this proposal, it seems natural that contextual drift should apply to the learner's ongoing neural activity.  Importantly, this has been considered with respect to memory models \cite<e.g.,>{SedeEtal2008} and demonstrated in analyses of EEG \cite{MannEtal2007,MannEtal2011}.  Therefore, this theory can be assessed by measuring the neural similarity between study repetitions as well as during test trials.

%%%%%%%%%%%%%%%%%%%%%%%%%%%%%%%%%%%%%%%%%%%%%%%%%%%%%%%%%%%%
% Study-phase retrieval
%%%%%%%%%%%%%%%%%%%%%%%%%%%%%%%%%%%%%%%%%%%%%%%%%%%%%%%%%%%%

\subsection{Study-Phase Retrieval}

% More study-phase retrieval papers \cite{LohnEtal2011,CepeEtal2009}.

% \citeA{ThioDAgo1976} proposed that the retrieval operation at the second presentation (or subsequent presentations) is similar to the retrieval operation at the time of subsequent recall. (ThioDAgo1976 also does lag effect)

Study-phase retrieval proposes that the repetition of an item retrieves the memories of its earlier presentation(s), including both item and contextual information, and that this retrieval during study is important for improving subsequent memory \cite{Gree1989a,ThioDAgo1976}.  Critically, the learner must recognize that the repeated item has been encoded previously in order to benefit from spacing \cite{HintBloc1973,HintEtal1975a,JohnUhl1976,Raai2003}.  Surely massed items will be recognized as being repeated as well, so why do the spaced items benefit more (the crux of the spacing effect)?  As in contextual variability, context varies more across spaced repetitions; a repetition is assimilated into the existing memory trace, and this provides an addition set of retrieval cues for spaced items.
% The inverted U-shaped spacing effect curve found with long inter-study lags occurs because
As lag increases between repetitions it becomes more difficult to retrieve the prior study event, but if a longer lag item is retrieved then the memory trace is strengthened to a greater extent \cite{DelaEtal2010}.  Notably, this retrieval difficulty idea is in line with research on the testing effect \cite<e.g.,>{KarpRoed2007}. \hl{(AH asked: how is it in line?)}

% Frequency judgments reveal that people notice when items repeat (Hintzman).

%%%%%%%%%%%%%%%%%%%%%%%%%%
% SPR question: how are memories stored
%%%%%%%%%%%%%%%%%%%%%%%%%%

This theory brings up the question of how the memories are stored.  Is a recognized repetition stored as a new trace, or is the original trace strengthened?  As alluded to above, \citeA{Raai2003} explained that a repetition should strengthen (and add to) the initial stimulus trace.  If each repetition was instead stored as a separate trace, the first trace will decay at longer lags (effectively a long retention interval) and subsequent memory may rely solely on the second trace, a potential issue for the contextual variability theory.  However, this is not the typical result of a spacing manipulation, except at very long lags as \citeA{Glen1976} explained.  Updating an existing trace with new information is implemented in others models as well \cite<e.g.,>{HowaKaha2002,PolyEtal2009}.
%\hl{Would this predict that the first trace should be recalled (better than the second trace) under deliberate encoding variability?}

%%%%%%%%%%%%%%%%%%%%%%%%%%
% SPR: reinstatement during study
%%%%%%%%%%%%%%%%%%%%%%%%%%

Assuming that a study repetition brings to mind its prior occurrence(s), study-phase retrieval theory predicts the reinstatement of neural context \cite{LohnKaha2014b}.  Specifically, the neural activity during a repetition A' will be more similar to its initial presentation A (due to study-phase retrieval) than to the items just prior to the repetition A', given a spaced trial.  This is different than what would be expected under the contextual variability theory where context should simply drift (A' would be more similar to the items just prior to it than to presentation A).  Along these lines, \citeA{TurkEtal2012b} showed that during visual scene processing, an identical scene presented on different lists but preceded by the same stimulus (i.e., the same context) was found to evoke more similar neural activity compared to when it was preceded by different stimuli (supporting study-phase retrieval).
% Because similarity may be influenced by low-level perceptual features, it may be possible to make comparisons using neighboring trials after retrieval is assumed to have taken place.

%%%%%%%%%%%%%%%%%%%%%%%%%%
% SPR: reinstatement during test
%%%%%%%%%%%%%%%%%%%%%%%%%%

Contextual reinstatement is also thought to occur during a memory test.  Some models assume that the current contextual state is used as a cue for retrieval attempts, especially during a free recall test without experimenter-supplied cues.  When an item is recalled the present context is updated with that item's associated context \cite{HowaKaha2002,SedeEtal2008}. This assumption helps account for particular patterns of recall in experimental evidence \cite{Kaha1996,LohnKaha2014b}.
Additionally, evidence for similar neural activity during study and recall of a given stimulus shows that this kind of contextual reinstatement occurs in the brain \cite{MannEtal2011,PolyKaha2008,XueEtal2010}.

%%%%%%%%%%%%%%%%%%%%%%%%%%%%%%%%%%%%%%%%%%%%%%%%%%%%%%%%%%%%
% Hypothesis combinations
%%%%%%%%%%%%%%%%%%%%%%%%%%%%%%%%%%%%%%%%%%%%%%%%%%%%%%%%%%%%

\subsection{Hypothesis interactions}

It is important to note that these theories are not mutually exclusive or necessarily competing, and it would be difficult (if not impossible) to test each independently.  In fact, they can work well together, and as described above the consensus in the literature is that a hybrid account is needed to explain spacing effects \cite{DelaEtal2010,Gree1989a,LohnKaha2014b,LohnEtal2011,Raai2003}.
%memory models that account for the spacing effect do typically integrate more than one mechanism.
% As an additional example, in TCM \cite{HowaKaha2002}
Most hybrid accounts agree that each presentation is encoded with drifting context (contextual variability) and that an item repetition is assimilated with prior occurrences (study-phase retrieval).
% This idea of the interaction between contextual variability during encoding and retrieval and the reinstatement of context giving rise to the spacing effect is assumed in other popular theories and models \cite{Gree1989a,SedeEtal2008} and
This combination seems necessary to account for the effect when fitting models to empirical data \cite<e.g.,>{LohnKaha2014b,MozeEtal2009,PavlAnde2005}.
% \cite[p.~8]{LohnKaha2014b}
% When item repetitions occur during study,

%%%%%%%%%%%%%%%%%%%%%%%%%%%%%%%%%%%%%%%%%%%%%%%%%%%%%%%%%%%%
% Mathematical models
%%%%%%%%%%%%%%%%%%%%%%%%%%%%%%%%%%%%%%%%%%%%%%%%%%%%%%%%%%%%

\subsection{Formal models accounting for the spacing effect}

An important way to constrain explanations of the spacing effect is with the mechanisms implemented in computational models that can account for patterns of human performance.  Most of these models contain interacting mechanisms that involve more than one of the hypotheses described here.

\citeA{Raai2003} made an influential model based on the Search of Associative Memory model \cite<SAM;>{MensRaai1989} using the spacing effect theory proposed by \citeA{Glen1979}.  It accounts for the spacing effect in cued recall using contextual variability and study-phase retrieval mechanisms.

An activation-based account of the spacing effect was implemented by \citeA{PavlAnde2005} in ACT-R, and can explain the effect at multiple timescales, which the SAM-based model cannot.  Here, all repeated items
% (massed and spaced)
receive a strength increment, but the rate of decay for the resulting trace is positively correlated with the level of activation for that item at the time of its repetition.
% If an item has high activation when it is repeated (as would be expected for a massed item) then it will decay quickly, whereas low activation at a repetition (for a spaced item) would lead to slow decay of the trace.
This leads to a spacing effect that fits their behavioral data well.  The quick decay for a massed item is reminiscent of deficient processing, but the mechanism is not explicitly defined in this way.

The Multiscale Context Model \cite<MCM;>{MozeEtal2009} can also account for the spacing effect at multiple timescales.  It makes impressively accurate predictions at various inter-study lags and retention intervals, as well as for different study materials, using relatively few parameters.  It implements contextual variability and study-phase retrieval (``retrieval-dependent update'') in a method similar to the SAM model for storing and retrieving context and item information.  Additionally, it uses a mechanism similar to the ACT-R model's decay to predict a forgetting function.

% % MCM details
%, which is described as a predictive-utility theory.  A basic interpretation of this mechanism is that for a given lag between item repetitions, once that lag length has passed without encountering the item again, the ``need'' for the model to remember the item decreases and the memory trace can decay.
% Therefore, a massed repetition should decay relatively quickly while a spaced repetition should lead to a more persistent trace.
% need probability is a positively correlated function of the repetition lag

The Context Maintenance and Retrieval model \cite<CMR;>{LohnKaha2014b,PolyEtal2009} can account for the spacing effect in free recall. As its name implies, contextual variability is an essential aspect for modeling the encoding of context as it fluctuates, and a study-phase retrieval mechanism helps it reinstate previous contextual states.  These two aspects allow it to capture patterns of spacing effect results in a paradigm that the other models were not designed to fit.

Overall, contextual variability and study-phase retrieval are clearly important theories, as these mechanisms are central to successful models that capture the spacing effect.  Even though none of these models use deficient processing, it should be given attention due to its recent popularity in empirical investigations involving neural recordings.

%%%%%%%%%%%%%%%%%%%%%%%%%%%%%%%%%%%%%%%%%%%%%%%%%%%%%%%%%%%%
% Previous EEG studies
%%%%%%%%%%%%%%%%%%%%%%%%%%%%%%%%%%%%%%%%%%%%%%%%%%%%%%%%%%%%

\subsection{EEG studies of the spacing effect}

\citeA{VanSEtal2007} published the only study investigating the spacing effect using event-related potential (ERP) and time--frequency analyses, which was based on previous stimulus repetition research.  They used a continuous recognition paradigm (repeatedly judge word presentations as being new or old) followed by a surprise recall test.
Unfortunately, some of their EEG effect interpretations do not agree with episodic memory research and instead are explained as being specific to working memory paradigms like continuous recognition, which does not seem ideal to base new research on.


Although the behavioral results of \citeA{VanSEtal2007} indicate higher recall for spaced \textit{vs.} massed repetitions, they did not analyze neural data for an interaction between spacing and memory which seems like an important factor to investigate in a study of the spacing effect.  Also, their use of continuous recognition may have confounded their interpretation of the data because this paradigm induces a testing effect (every trial is an old/new test).  Behavior may differ qualitatively compared to simply studying and encoding stimuli at spaced and massed intervals \cite[p.~91]{DelaEtal2010}.
Thus, further research is needed to understand the EEG patterns and cognitive processes that underlie the spacing effect.  Despite these shortcomings, their results can be used as a basis for future experiments to make effect predictions when modulated by subsequent memory, an analysis that has not been previously explored using neural data.

% Importantly, what is truly meant by \emph{the spacing effect} is only captured in an analysis that depends on how the spacing of study events affects memory performance at a subsequent point in time (a spacing $\times$ subsequent memory interaction).


%%%%%%%%%%%%%%%%%%%%%%%%%%%%%%%%%%%%%%%%%%%%%%
% BONEYARD!
%%%%%%%%%%%%%%%%%%%%%%%%%%%%%%%%%%%%%%%%%%%%%%

%%%%%%%%%%%%%%%%%%%%%%%%%%
% Van Strien et al. (2007) details
%%%%%%%%%%%%%%%%%%%%%%%%%%

% (seen for the first time or seen earlier in the list)
% Since no prior study used subsequent memory tests, which is a necessity for assessing the spacing effect, they implemented a surprise free recall test.

% They performed two types of analyses using two ERP effects (the N400 and LPC) as well as theta and upper alpha power.  There were two main analyses: (1) they compared EEG responses for correctly recognized new and old words; there was an N400 attenuation for massed (denoting less semantic processing) compared to spaced and an enhanced LPC for massed (better matched based on memory strength) compared to spaced.  There was also increased theta for massed (more working memory engagement) and decreased alpha for spaced (more long-term retrieval), though these interpretations do not exactly agree with episodic memory research and instead are explained as being specific to working memory paradigms like continuous recognition.  (2) They examined correlations between recall performance and the EEG effects.  Higher LPC voltage and higher theta power were both associated with lower recall indicating that a better match to the contents of working memory leads to less processing which in turn leads to worse subsequent memory.  There were no significant correlations with the N400 or alpha power.  Overall, these results mostly support the deficient processing theory.

% Note that the LPC is similar in timing and topography to the parietal old/new effect, an ERP component related to recollection \cite{Curr2000,RuggCurr2007,Wild2000}.  The authors make the distinction that the latter effect is associated with recollection while the LPC is related to the P300; we will use the LPC term when replicating their results.

% Other relevant ERP and time--frequency effects will be described in the next section (Background and Significance).

% Importantly, what is truly meant by \emph{the spacing effect} is only captured in an analysis that depends on how the spacing of study events affects memory performance at a subsequent point in time (a spacing $\times$ subsequent memory interaction).  The first analysis described above is simply a recognition test with a repetition lag difference and cannot speak to the cause of \emph{the spacing effect}.  The second analysis hints that there are interesting effects to explore based on subsequent memory performance, but they did not do this.   While the behavioral results of \citeA{VanSEtal2007} indicate a recall advantage for spaced repetitions (higher recall for spaced \textit{vs.} massed repetitions), they unfortunately did not analyze neural data for the critical interaction.

% A final point to consider is that their use of continuous recognition may have confounded their interpretation of the data because this paradigm induces a testing effect (every trial is an old/new test).  Participants may show qualitatively different behavior here compared to simply studying and encoding stimuli at spaced and massed intervals \cite[p.~91]{DelaEtal2010}.
% Thus, further research is needed to understand the EEG patterns and cognitive processes that underlie the spacing effect.  Despite these shortcomings, their results can be used as a basis for future experiments to make effect predictions when modulated by subsequent memory, an analysis that has not been previously explored using neural data.

%%%%%%%%%%%%%%%%%%%%%%%%%%
% ERP/TF effects
%%%%%%%%%%%%%%%%%%%%%%%%%%

% \subsection{EEG effects in relation to memory and attention}

% Some ERP effects were mentioned briefly in relation to \citeA{VanSEtal2007}. These are the N400 which is related to \hl{semantic processing}, and the LPC which is related to template matching (a current stimulus matches a stored memory); these interpretations are apparently specific to continuous recognition paradigms, but may occur in paradigms involving repetition.  In addition to replicating their analyses, other predictions can be made for the proposed EEG analyses.
% Early ERP components have been related to attentional processing
% (e.g., the visual N1) \cite{LuckEtal2000}, and this seems like a component that may show a spacing effect.
% % (e.g., the posterior N2 and P300) \cite{FolsVanP2008,WagnEtal1999}.
% Under the deficient processing hypothesis, less attention is paid to massed compared to spaced repetitions, so we would expect amplitude decreases to occur for massed repetitions.


% ERP effects in the same time and electrode regions could also be related to successful encoding \cite{FrieEtal1996,PallEtal1988}, but may be dissociable from parietal old/new effects based on subsequent memory analyses (e.g., spaced items are remembered better than massed).  However, since \citeA{VanSEtal2007} claim that the LPC (and N400) should be interpreted differently in a continuous recognition paradigm, we also may not see these effects with relatively short time intervals between repetitions.


% % TC: Might be worth adding that the FN400 is less like to be affected by attention fluctuations during encoding.  http://psych.colorado.edu/~tcurran/Curran_2004.pdf   On the other hand, the other accounts may predict familiarity differences?  Any other recollection/familiarity data out there on the spacing effect?

% % Prediction: Massed forgotten should have the largest parietal old/new effect. Spaced remembered the smallest, but still an effect compared to first or single presentation items.

% The analysis of neural oscillations has become an important tool for investigating episodic memory \cite<for reviews, see>{HansStau2014,NyhuCurr2010}.  Theta power (4--7~Hz; frontal, temporal, and parietal) is related to memory formation and retrieval, particularly in medial temporal lobe regions (\citeNP{Klim1999,KlimEtal1996b,KlimEtal2006,LongEtal2014a,SedeEtal2003}; for a review, see \citeNP{MitcEtal2008a}).  It is also thought to reflect item--context binding \cite{HansEtal2009a,HansEtal2011a,StauHans2013,SummMang2005}.  Theta power should therefore increase for spaced compared to massed repetitions, but for different reasons under the hypotheses.  Under deficient processing, theta is decreased for massed repetitions simply because the item is not being processed well.
% Under both contextual variability and study-phase retrieval, theta would increase for spaced repetitions because the intervening context is also encoded (new information, item--context binding).
% However, there is a difference between CV and SPR; given that a repetition is properly re-encoded, in comparison to the initially encoded memory (Presentation 1), the ``contents'' of the two encoding events will be more similar under SPR and more variable under CV.
% % Maybe the similarity difference will only be in the theta band?


% Activity in the alpha band (8--12~Hz) is related to attention.  Lower alpha (8--10~Hz) is typically widespread across the scalp, and desynchrony (decrease in power) is related to increased attention while synchrony is sometimes associated with inhibiting task irrelevant cortical processes \cite{Klim1999,KlimEtal2007,PfurLope1999}.
% Upper alpha (11--12~Hz) desynchronization is more topographically restricted and is related to search and retrieval in semantic long-term memory, particularly in regions relevant for stimulus integration \cite{Klim1999,KlimEtal2005}.  The combination of upper alpha desynchronization and a theta synchronization has been shown to relate to effective learning of pairs of words over left frontotemporal regions and pairs of faces over right parietal regions \cite{MollEtal2002}.  Decreases in power in the lower beta band (13--21~Hz; central and temporal) are similarly associated with increased semantic encoding \cite{HansEtal2011a,HansEtal2012}.  Under the deficient processing hypothesis, we would expect more desynchronization in all three bands for spaced compared to massed repetitions because processing is reduced for massed repetitions.

% Additionally, if a massed representation is still active during repetition \cite{Chal1993}, it may be the case that semantic processes should come online earlier than for spaced repetitions.  The exact synchronization pattern that follows would inform the hypotheses differently.

% HansEtal2011a, p. 15679 - semantic first, then item--context binding.

%%%%%%%%%%%%%%%%%%%%%%%%%%
% DP: explanatory details
%%%%%%%%%%%%%%%%%%%%%%%%%%

%%%%%%%%%%%%
% DP: extra info probably not needed here
%%%%%%%%%%%%

% On the other hand, when repetitions are spaced and the item is relatively less familiar at the second presentation (i.e., is no longer in working memory and needs to be retrieved from long-term memory), more attention is paid compared to the massed condition.  The locus of the spacing effect under the deficient processing account must be either between two presentations (and therefore a function of rehearsal \citeNP{Gree1989a}) or at the second presentation (and therefore a function of attenuated attentional processes), or both \cite{Hint1974}.

%%%%%%%%%%%%
% DP: random notes
%%%%%%%%%%%%

% \cite{DelaEtal2010} also has a good review that includes extension to \cite{Chal1993}.

% \cite{ToppEtal2009} cited by \citeA{LohnKaha2014b}. The strength of retrieved information is inversely proportional to its lag \cite{PavlAnde2005}.

% A priming study by \citeA{HensEtal2004} examined both ERPs and fMRI BOLD activity to repeated stimuli presented at four different lags (ranging from zero to tens of stimuli).

% Bjork & Bjork, page 32?-331
%
% http://books.google.com/books?hl=en&lr=&id=f7p_nbo40IAC&oi=fnd&pg=PA317&dq=%22distributed+practice%22+%22cued+recall%22&ots=kz_S1c84eu&sig=bAl7zdUMNqxc99RouaiYtoyw7Jc#v=onepage&q=%22distributed%20practice%22%20%22cued%20recall%22&f=false
%
% http://books.google.com/books?id=f7p_nbo40IAC&printsec=frontcover#v=onepage&q&f=false

%%%%%%%%%%%%
% DP: Predictions
%%%%%%%%%%%%

% IS THIS CORRECT? A retrieval manipulation was shown to eliminate the spacing effect by \citeA[Experiment 1]{Glen1979}, in which they varied the presentations in a cued recall test
% Because the spacing effect can be eliminated through varied presentations \hl{(how did they do it? varied pairing of stimuli?)}, and thus that varied presentations must eliminate deficient processing (because more attention is paid??), this theory would predict that... \cite[Experiment 1]{Glen1979}.
% \cite[p.~110]{Glen1979}

% % Deficient processing
% \subsubsection{Predictions}

% \begin{itemize}
%     \item As inter-study lag increases, deficient processing should decrease.

%     %\item EEG: Decrease in ERP attentional components for massed vs spaced repetitions \cite<e.g., posterior N2>{FolsVanP2008}.

%     \item EEG: Decrease in ERP attentional components for massed vs spaced repetitions \cite<e.g., visual N1>{VogeLuck2000}.

%     \item EEG: Increased alpha (8--10~Hz) for massed vs spaced repetitions, indicating less attentional processing \cite{KlimEtal2007}.

%     \item Similarity: Decreased similarity for massed repetitions vs spaced, due to less processing of the massed repetition.

%     \item Classification: Lower accuracy for massed repetitions vs spaced, due to less processing of the massed repetition.

%     %\item Deficient processing due to neural repetition suppression should be eliminated as long as repetitions are not immediate (i.e., no deficient processing should occur at either short or long lag repetitions). This should not differ for different lags.
% \end{itemize}

%%%%%%%%%%%%%%%%%%%%%%%%%%
% CV: explanatory details
%%%%%%%%%%%%%%%%%%%%%%%%%%

%\citeA[p.~9]{LohnKaha2014b} \hl{has a good review of this}.
%\citeA[p.~92]{DelaEtal2010} \hl{discusses} \citeA{BrauRubi1998}.

% Explain OR Scores \cite{RossLand1978,LohnKaha2014b,LohnEtal2011}.

% \hl{(Undeveloped paragraph:)} It has been shown that particular patterns of neural activity are indicative of successful encoding.  Because the contextual variability explanation of the spacing effect relies on a comparison of the similarity between the test context and that of the study episode, it might be thought that this hypothesis would not be able to explain SME analyses that rely only on the encoding period.  However, explanations of the SME comparisons cannot rely on only analyzing the encoding period; subsequent memory performance is the product of many factors, including the processes active during encoding, the type of test used, the levels-of-processing match between study and test, and the retention interval between study and test, among others. So, contextual variability takes both encoding and retrieval processing into account, and the processes at each stage are intimately tied together.

% It should be noted that it seems unlikely that contextual variability during encoding of repetitions is the sole reason for the spacing effect because of the results of the lag effect. Specifically, spaced words presented at short and long lags should be encoded with the same number of cues (as opposed to massed presentations), and so under this theory longer lags would not have better subsequent memory than shorter lags.

% Figure out how to summarize this: ``If the retention interval is short, closely spaced practices will be remembered better than widely spaced practices because the testing context will be similar to all the contexts of the closely spaced practices, but it will be similar to only the contexts of the most recent widely spaced practices. In contrast, if the retention interval is long, closely spaced practices will result in poorer recall because the test context will have fluctuated away from the overlapping encoding contexts, whereas widely spaced practices will result in better recall because the more diverse contextual information encoded will be more likely to match the test context.'' \cite{PavlAnde2005}

%%%%%%%%%%%%
% CV: Study--test match
%%%%%%%%%%%%

% Memory performance is better when the retrieval context is similar to that of the study episode.  Specifically, the reinstatement of a previous context at test (that is, the similarity of the test context to that of the study episode is increased)
% % be it conscious or unconscious
% results in retrieval facilitation for information encountered in that previous episode.  The contextual variability hypothesis posits that the spacing effect occurs because in the spaced condition the two study contexts are less similar to each other and therefore encode more contextual features than in the massed condition where the two study contexts are more similar to each other \cite{Glen1976,Glen1979}.  At retrieval, there is a higher probability that the test context will be similar to one of the spaced contexts compared to the massed context(s); this is because there are more potential retrieval cues that might reinstate the study context from (one of) the spaced condition study episodes (i.e., two chances instead of one).

%%%%%%%%%%%%
% CV: Predictions
%%%%%%%%%%%%

% % Contextual variability
% \subsubsection{Predictions}

% % Presentation-1 and Presentation-2 should be less similar in spaced as compared to massed.

% % Contextual variability: (more contextual drift, better chance for memory retrieval; could also be due to encoding the stimulus differently each time):

% \begin{itemize}
%     \item The spacing effect occurs because spaced repetitions have more associated contextual retrieval cues than massed items.
%     %\item Behavioral: Using multiple lags ($>0$), items with longer lags should be remembered better than items with shorter lags due to the composite trace containing more retrieval cues in the longer lag conditions.
%     % is this the same prediction as deficient processing decreasing?


%     \item Similarity: Evidence for (neuro)contextual drift across trials \cite<e.g.,>{MannEtal2011} would provide a basis for making predictions about the expected similarity between massed repetitions and between spaced repetitions. The prediction is that similarity decreases with inter-study lag.
%     \begin{itemize}
%         \item One method to assess this is with neural pattern similarity autocorrelations across contiguous trials, as in \citeA{MannEtal2011}.
%         \item Another is to compare the similarity of once-presented items with nearby and distant items (because these trials will not be affected by study-phase retrieval).
%     \end{itemize}

%     \item Under these assumptions, memory performance should be negatively correlated with similarity (i.e., better memory for less similar items).

%     %\item Can augment similarity analysis by only looking at similarity of trials that were correctly classified as showing face or house patterns. This would ensure that encoding was happening on every trial.

%     \item Classification: No accuracy difference predicted between spaced and massed items.

%     %\item Similarity: Greater similarity between Presentation-1 and Presentation-2 for massed vs spaced (or as lags get shorter), due to less contextual drift for massed.

%     %\item Similarity: Would be able to assess contextual drift without the influence of study-phase retrieval (described below) by assessing the similarity of once-presented items compared to nearby and distant items. This would then provide more support for the prediction of greater similarity for massed compared to spaced items.
% \end{itemize}

%%%%%%%%%%%%%%%%%%%%%%%%%%
% SPR: explanatory details
%%%%%%%%%%%%%%%%%%%%%%%%%%


%%%%%%%%%%%%
% SPR: Predictions
%%%%%%%%%%%%

% % Study-phase retrieval
% \subsubsection{Predictions}

% \begin{itemize}
%     \item EEG: Massed items are recognized (i.e., retrieved) as being more familiar, and thus should show larger recognition memory ERP effects (FN400 and Parietal Old/New).

%     \item Similarity: A study repetition is more similar to the initial presentation than to the item that precedes the repetition (inter-item associative context).  The item following the repetition could be used to escape the low-level perceptual feature confound because the brain will be in a contextual state that has not drifted much.
%     % \hl{(But what about reinstatement during that item?)}

%     \item Similarity: If reinstatement occurs during test, the test presentation should be more similar to the study presentations than to the prior test item (or other study items).

%     \item Similarity: Under study-phase retrieval, a larger variety of contextual elements gets integrated into the combined memory trace at a repetition for spaced compared to massed repetitions.  The joint similarity between the repeated study presentations and the test should be more similar for spaced compared to massed practice because there are more chances for the test context to be close to that of the study presentations for spaced repetitions.  \hl{(Note to myself: Analyzing this kind of joint similarity will need to be thought out. I may be able to use the exact same methods, or I may need to use a matlab function like linkage.m.)}
%     \begin{itemize}
%         \item This differs from the contextual variability hypothesis because SPR predicts more similarity between the spaced study presentations while CV predicts less similarity; likewise, at test SPR involves contextual reinstatement (more of an active update of context) while CV requires a match with some contextual element from study.
%     \end{itemize}

%     \item Similarity: If subsequent memory performance benefits from study-phase retrieval, then the study presentations of a remembered item should have greater similarity to each other than a forgotten item (regardless of spacing).
%     %\item Classification: Along similar lines, higher accuracy for subsequently remembered vs forgotten items (regardless of spacing). % don't understand this
%     \begin{itemize}
%         \item Since spaced items are remembered better on average, the above effect should be stronger for spaced items (more ``signal'' in the encoded trace).
%     \end{itemize}
%     \item Classification: Along similar lines, classification accuracy will be better when there is more item ``signal'' in the trace, so this will be a stronger effect for both remembered and spaced items. This should happen at both a study repetition and test.

%     \item Behavioral: In a free recall test, if a spaced item is presented at positions $i$ and $j$, likely to recall items $i+1$ and $j+1$ together because they are associated with the same context \cite<shown by>{LohnKaha2014b,PolyEtal2009}. This would speak to the idea that a prior contextual state can be reinstated, and that this influences recall.
%     \begin{itemize}
%         \item Similarity: Neural patterns related to above prediction should be observed. Greater similarity during recall of items $i+1$ and $j+1$ because they followed similar contextual states during study. \hl{(Is this reaching too far?)}
%     \end{itemize}

%     %\item Similarity: Greater Presentation-1--Presentation-2 similarity for recalled spaced vs massed items, due to reinstatement of Presentation-1 during Presentation-2.
% \end{itemize}

%%%%%%%%%%%%%%%%%%%%%%%%%%
% Interactions: explanatory details
%%%%%%%%%%%%%%%%%%%%%%%%%%


%%%%%%%%%%%%
% Interactions: Predictions
%%%%%%%%%%%%

% However, behavioral and electrophysiological predictions can be made for each theory and for their combinations, and these can be assessed using an appropriate spacing effect paradigm.  For example, if information is retrieved during a study repetition (or during test), a massed repetition will have greater similarity to its prior occurrence as compared to a spaced repetition because they were presented close together in time.  Does this interact with contextual variability; i.e., is an item remembered better when its repetition is less similar to the initial presentation?  Perhaps this is indicative of thinking about the stimulus in a different way.


% Many questions still exist regarding temporal context in relation to the spacing effect.  Is context reinstated on a study repetition, and is it modulated by whether the episode was massed or spaced?
% % (Do we see evidence of the Presentation-1 picture context at Presentation-2 word-only?)
% What is the prediction for the similarity of repeated study contexts for spaced vs massed repetitions?

%What is the prediction for the similarity between the test context and the study context for spaced vs massed?

% % interactions
% \subsubsection{Predictions}

% \begin{itemize}
%     \item SPR $\times$ CV: Behavioral: Is it possible to improve subsequent memory by inducing similar contexts through the presentation of the same two stimuli prior to each repetition of an item \cite<study-phase retrieval;>{TurkEtal2012a}?  Or would this lead to worse subsequent memory due to a smaller variety of retrieval cues being encoded (contextual variability)?  This might depend on definition of context (e.g., is there a drifting internal context that is unaffected by the outside world?).
%     \item SPR $\times$ CV: Similarity: Induced context would predict greater similarity between presentations of an item.  This can only be tested with spaced items.

%     \item SPR $\times$ DP: Similarity: Greater study presentation similarity for recalled spaced vs recalled massed items, due to effort to reinstate initial context during a spaced but less processing for massed.
% \end{itemize}

% \subsection{Study condition predictions}

% \subsubsection{Study conditions controlling for prior context}

% Same context: Presentation-2 of word is preceded by the same-modality image as on Presentation-1.

% Varied context: Presentation-2 of word is preceded by the other-modality image as on Presentation-1.

% \subsubsection{More extreme}

% Same context: Presentation-2 of word is preceded by the same word--image pair as on Presentation-1.

% Varied context: Presentation-2 of word is preceded by a different word--image pair as on Presentation-1.

% A1 B2 C3 D4 E5 F6 G7 H8 I9

% H8 D4 E5 B2 C3 F6 G7 A1 I9

% \subsection{Analysis methods}

% \subsubsection{EEG}

% Relate ERP and TF effects in the literature to particular cognitive processes.

%%%%%%%%%%%%%%%%%%%%%%%%%%
% EEG is useful because...
%%%%%%%%%%%%%%%%%%%%%%%%%%

% \subsection{Why EEG is useful}

% EEG is an important methodology when investigating encoding and
% retrieval processes because of its high temporal resolution, and it
% can reveal the unfolding of cognitive processes across brief periods
% of time.  Additionally, over the past few years simple and relatively
% inexpensive commercial EEG systems have become available to the
% public.  This availability will likely fuel research (and industry)
% based around improving learning techniques that will evolve out of
% credible cognitive neuroscience-backed research on memory encoding and
% retrieval.  Thus, the practical impact that this research could have
% is directly related to the industry's emerging use of EEG products as
% well as public availability (ubiquity?) of personal devices that
% monitor biological data.

%%%%%%%%%%%%%%%%%%%%%%%%%%
% Pattern classification is useful because...
%%%%%%%%%%%%%%%%%%%%%%%%%%

% \subsection{Why pattern classification is useful}

% Understanding the differences between successful and unsuccessful
% memory formation is a useful endeavor.  However, in the SME
% literature, traditional statistical analyses are based on averaged
% data, so it is nearly impossible to determine whether learning
% occurred on any particular trial.  Also complicating the task of
% determining whether successful encoding occurred is that participants'
% memories must be assessed at a later point, which is a time consuming
% process when trying to improve learning effectiveness.  Clearly
% univariate statistical analyses, which essentially look for amplitude
% differences between conditions in the psychological contrast of
% interest, can only take us so far, as there are effects hidden in
% patterns of neurophysiological data that are undiscoverable by these
% types of analyses.  These patterns of activity are a proxy for the
% mental or cognitive representations occurring in our brains, and we
% can use more powerful and nuanced statistical methods to extract and
% examine the information contained in them.

% If we understand both the commonalities across different successful
% encoding situations and the important features of specific encoding
% contexts, it might be possible to predict with relatively high
% likelihood whether information was encoded.  Once an accurate model
% has been established (e.g.,~a particular set of patterns are seen
% during successful encoding when the information is later retrieved in
% a certain way) then that model can be used to assess future trials.
% Alternatively, perhaps if the neural patterns resulting from different
% learning and retrieval conditions for different types of stimuli were
% sufficiently characterized, it would be possible to identify those
% patterns that are indicative of learning under specific conditions.
% An interesting direction to take this formidable endeavor would be to
% use machine learning techniques to discover when learning occurred.

% Multivariate statistical methods, such as pattern classification, are
% useful because they can reveal effects that univariate statistics do
% not find.  \citeA{WataEtal2011} used this technique using fMRI, giving
% the first demonstration of classifying neural data based on future
% memory performance.  The authors used activation patterns in the
% hippocampus and parahippocampal cortex from one half of the encoding
% periods (categorized by subsequent memory) to accurately predicted
% subsequent memory performance in the second half of encoding periods.
% The ability to use machine learning to classify single trials based on
% their subsequent memorabilities is an incredible feat and could have
% huge implications for increases in real-world learning efficiency.
% \citeA{LeeEtal2012} hypothesize that the multivariate approach has
% increased sensitivity over univariate analyses, and this allows them
% to find effects in the former and not the latter.  They then discuss
% how \citeA{WataEtal2011} did not find univariate effects, but were
% able to use support vector machines (SVMs) to predict subsequent
% memory, and mention a few other studies in similar situations
% \cite{KuhlEtal2012b,XueEtal2010}.  The power of these multivariate
% approaches is revealed in their results, and can be described thusly:
% finding differences using classifiers is more about the patterns of
% neuronal activity across space and time (and the information contained
% in the signal), rather than differences in the absolute levels.


%%%%%%%%%%%%%%%%%%%%%%%%%%
% Other
%%%%%%%%%%%%%%%%%%%%%%%%%%

% The study-phase retrieval theory works well together with the contextual variability theory. An item repetition (that is recognized as having been seen on a prior occasion) will not simply lead to the encoding of a new memory trace that includes the current context, but instead will assimilate the present context with that of all prior occurrences of the item into the trace \cite{Gree1989a,HowaKaha2002,LohnKaha2014b,SedeEtal2008}. Thus, the encoded representation of a stimulus repetition will be more similar to its prior occurrences than to nearby unrelated items. For example, during visual scene processing, identical scenes presented on different lists but preceded by the same stimuli (i.e., the same context) were found to evoke more similar neural activity compared to when they were preceded by different stimuli \cite{TurkEtal2012b}.

% Interestingly,  \citeA{LohnKaha2014b} found that spacing effects in a free recall experiment must be the result of more than just a contextual variability mechanism (p.~8). It is the interaction between study-phase retrieval and contextual variability that gives rise to the spacing effect.

% While some researchers define context as the experimental backdrop in which items are presented \cite{DelaEtal2010,Este1955a}, \citeA{HowaKaha2002} and \citeA{SedeEtal2008} define context as being intimately related to the items being studied and its fluctuation is driven by those items. This predicts item contiguity effects, which are readily seen in studies of free recall \cite<e.g.,>{Kaha1996}.

