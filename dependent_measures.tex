% !TEX root = ./diss.tex

\chapter{Dependent measures}

The terms subsequent memory and subsequent retrieval were mentioned earlier.  To be clear on these and related terms, subsequent memory effects (SMEs) involve analyses of the encoding phase and show differences that are contingent upon later memory performance.  Examining these differences sheds light on why some things can be remembered later while others are forgotten.
Additionally, the retention interval between the final study event and the test event will also have an impact on the optimal lag \cite{CepeEtal2006,Glen1976,Glen1977,Glen1979}; however, optimal lag is not considered here.

We note that it is possible that SMEs are actually due to subsequently remembered items having systematically different characteristics than forgotten ones and not because different cognitive processes were engaged that affected memory encoding differently.
% This would be an issue for basically every EEG or neuroimaging study examining SMEs or old/new recognition effects (which examine differences during the test phase).
Though this is a potential issue for the present experiments in relation to SMEs, this is a relatively common analysis in the memory literature due to the interesting nature of the questions that it can inform.  Analyses like this are typically done without addressing or even acknowledging possible item selection effects, and this caveat would apply to most EEG and neuroimaging studies examining SMEs or old/new recognition effects (which examine differences during the test phase).
Because word and image stimuli were randomly assigned to conditions for each participant, item differences are not a factor for independent variables that were directly manipulated.

Because the spacing effect predicts a long-term memory advantage for spaced compared to massed study repetitions, it seems likely that there will be differences between spaced and massed repetitions that interact with subsequent memory.  However, it also seems likely that when encoding is successful for both spaced and massed repetitions, the active memory mechanisms should be qualitatively similar.  There may be a quantitative difference (the difference is in the degree of processing), but it remains possible that memory will not be critical to determining why the spacing effect occurs; many factors besides encoding mechanisms can influence subsequent memory.  Although we will investigate neural patterns known to be associated with memory encoding and retrieval, other cognitive factors must be considered.  For example, attention and semantic processing likely influence the spacing effect.  Effects of spacing for repetition events without a subsequent memory interaction can also reveal processing differences between spaced and massed items.


\section{Neural data}

% The present experiments investigate the cognitive processes and neural substrates by examining recordings of the brain's electrical activity
% (the electroencephalogram, or EEG).  When the EEG is segmented into
% brief windows of activity that are time-locked to an event such as the
% presentation of stimuli, these epochs are called event-related
% potentials (ERPs).  In relation to a psychology experiment, ERPs are
% subdivided into conditions based on specific criteria and are averaged
% within these conditions to show the common voltage deflections, or ERP
% components, in the EEG signal associated with each condition.
% Comparisons between conditions can help determine when and where cognitive processes change across conditions.

EEG recordings provide a fine-grained time course of the electrical
activity of neurons, on the order of milliseconds, which is important
for determining when neural processes occur with respect to behavioral
responses.  EEG recorded at the scalp, as in the present experiments, represents the combined activity of millions of neurons.  For analysis, EEG is typically averaged across brief
epochs that are time-locked to events such as stimulus presentations,
and these event-related potentials (ERPs) are compared between
conditions to show differences in voltage deflections, known as ERP
components.  Particular ERP components, which dissociate in time
(relative to stimulus presentation) and space (at particular
electrodes), have come to be associated with different cognitive processes.

Similarly, decomposing EEG into time--frequency measures using methods such as wavelet transforms
%or multitapers
(like Fourier transforms, but they estimate the amplitude, or power, of
activity at a particular frequency across time) provides a more
nuanced (and higher-dimensional) representation of neural activity.
This methodology allows for an investigation of the oscillatory dynamics of neural
networks.  Oscillatory power is used to measure both local synchronous
activity and long-range communication between brain regions, and
different frequency bands have also come to be associated with particular
cognitive processes.  Power can be measured as a change relative to a pre-stimulus baseline, so when a frequency band increases in power (synchronizes) or decreases in power (desynchronizes), this means that the amplitude in that frequency band was higher or lower compared to the baseline.

%%%%%%%%%%%%%%%%%%%%%%%%%%
% Attentional effects
%%%%%%%%%%%%%%%%%%%%%%%%%%

\section{Effects of attention and semantic processing}

% This goal assesses how attentional and semantic processes are modulated by spaced and massed presentation, and mainly speaks to the deficient processing account.

Assessing how attentional and semantic processes are modulated by spaced and massed presentation is central to the deficient processing hypothesis.
% Though deficient processing implicates a deficiency for encoding (and subsequently remembering) massed items as opposed to an enhancement for spaced items \cite{DelaEtal2010}, it is worthwhile to investigate because recent research has given this hypothesis considerable attention \cite{BrauRubi1998,CallSchw2010,VanSEtal2007,WagnEtal2000,XueEtal2011}.
The present understanding is that an involuntary decrease in stimulus processing occurs for repetitions of recently encountered (massed) items, whereas this decrease does not occur for spaced repetitions.
% Along these lines, functional magnetic resonance imaging (fMRI) studies that involve stimulus repetitions have embraced the idea that neural repetition suppression may lead to the spacing effect \cite{CallSchw2010,WagnEtal2000,XueEtal2011}.

Under the deficient processing hypothesis, massed repetitions should involve decreases in attentional processes.  Because early ERP components have been related to attentional processing \cite{LuckEtal2000}, these may show spacing effects.  The visual N1 ERP component (early negative occipitoparietal peak) shows effects of selective attention \cite{KlimEtal2004} particularly during discrimination tasks \cite{VogeLuck2000}, and is related to increased visual analysis \cite{CurrEtal2002}.  It also shows effects of lexical and semantic processing \cite{ProvAdor2009}.
It does not typically show subsequent memory effects \cite<e.g.,>{CurrEtal2002,DuarEtal2006,DuarEtal2004}, and may act like an early attentional gating mechanism and attenuate for massed repetitions.

The analysis of neural oscillations has become an important tool for cognitive neuroscientists \cite<for reviews, see>{HansStau2014,NyhuCurr2010}.
Oscillatory desynchronization (decrease in power) in the lower alpha band (8--10~Hz; widespread) is related to increased attention \cite{Klim1999,KlimEtal2007,PfurLope1999}, though the functional interpretation of lower alpha is generally less clear than for other bands \cite{KlimEtal2007}.  Because deficient processing occurs for massed items, we would expect lower alpha to show more power (less desynchronization) for massed than spaced.

Another idea tied to deficient processing is priming, specifically semantic \cite{Chal1993} and perceptual priming \cite{MammEtal2002,RussEtal1998}.  A primed and still-active representation during a massed repetition will require less activation of semantic or perceptual features compared to spaced repetitions, and thus less processing will occur.
% is sometimes related to different effects depending on the paradigm, but in those involving repetition it
The N400 ERP component (negative central peak)
%is associated with the processing of meaning, and
is affected by semantic processing and priming \cite<for a review, see>{KutaFede2011}.  Performing a task involving semantic analysis will produce a more negative N400, linking it to the activation of semantic information \cite{KutaHill1980}.  The component is also linked to semantic anomalies, but in general, the N400 is thought to reflect the dynamic construction of semantic meaning as part of a larger distributed system (as opposed to being a direct measure of accessing the meaning of a stimulus).

N400 effects are seen in repetition paradigms where the component attenuates to repetition events compared to initial presentations \cite{KimEtal2001,OlicEtal2000,VanSEtal2007}; voltages becomes more negative as lag between repetitions increases.  The amount the component attenuates to a stimulus repetition or to a target stimulus following a semantic priming stimulus reveals ``how much of the information normally elicited by that stimulus is already active'' \cite[p.~23]{KutaFede2011}.  Spaced repetitions elicit an N400 (representing the activation of semantic processing) whereas this does not occur for massed repetitions because the information is primed and in working memory \cite{VanSEtal2007}.  This repetition attenuation should be a factor under deficient processing because failing to receive additional semantic processing events is one way subsequent memory could be worse for massed items.

% A frontal effect called the FN400 (similar to but different from the N400) is related to recognition memory, specifically in relation to whether this is familiar (i.e., old or new; the component attenuates for old items) \cite{Curr2000,Meck2006,RuggCurr2007}.  The FN400 is unlikely to be affected by attentional fluctuations during encoding \cite{Curr2004} and should simply differentiate old/new recognition.  Since participants are asked to make semantic associations in the present experiments, and because prior work has been done with the N400 in terms of stimulus repetition paradigms, we are going to interpret activity around 400~ms in terms of semantic processing.

% If massed repetitions are perceived as having stronger memory strengths compared to spaced repetitions (due to being primed and/or in working memory), retrieval mechanisms do not need to engage \cite{Gree1989a,VanSEtal2007}.  The FN400 ERP effect (negative frontal peak  indexes familiarity and should attenuate more for massed repetitions than spaced because massed will feel more familiar.

% \hl{(Tim asked: Or could it just be an FN400 that gets more negative with decreasing oldness? I asked Al about this during my proposal and he thought N400 seemed like a fine thing to call it.)}

For oscillatory effects, upper alpha (11--12~Hz; posterior) desynchronization (decrease in power) is related to the reactivation of semantic information from long-term memory \cite{Klim1999,KlimEtal2005}.  Decreases in power in the lower beta band (13--21~Hz; central and temporal) are similarly associated with the semantic processing of to-be-remembered items \cite{FellEtal2013,HansEtal2012,HansEtal2011a}.  If these bands reflect retrieving information from memory, we would expect to see less power in both for massed compared to spaced because the information is already primed and exists in working memory.  However, if the power decreases reflect the processing of semantic information after retrieval, both massed and spaced should show desynchronization, and massed repetitions may actually show an earlier onset because the information can be accessed faster.


%%%%%%%%%%%%%%%%%%%%%%%%%%
% Memory effects
%%%%%%%%%%%%%%%%%%%%%%%%%%

\section{Effects of memory retrieval and encoding}

%If attention and semantic processing are affected by study repetition lag, it follows that
Memory processes should also show effects of study repetition lag, as long-term memory performance is the critical measurement of the spacing effect.  This area of analysis
% assesses how activity related to memory processes is modulated by spaced and massed presentation.  It
can address all three theories, but provides a particularly important assessment of study-phase retrieval.


% \hl{(I mention the FN400 but never analyze it as a memory effect.)}

The late positive component/complex (LPC; also called the parietal old/new effect) is an ERP component that indexes conscious recollection.  Its amplitude correlates with the subjective amount of retrieved information \cite{VilbEtal2006,Wild2000,WildRugg1996} and is more positive for information that is subsequently remembered \cite{RuggEtal2002,WagnEtal1999}.
% It is similar in latency and topography to what is termed the late parietal component (LPC) by the only EEG investigation of the spacing effect \cite{VanSEtal2007}, but these authors state that it may be related to the P300 such that it reflects stimulus classification speed (``template matching'' to a representation stored in memory).
The LPC is also seen in short-term repetition experiments \cite{OlicEtal2000,VanSEtal2007},
% OlicEtal2000: 10-40s
and is linked to the conscious recognition of an item.
% Under deficient processing,
% both explanations should show the same pattern: whether the massed item has a stronger memory strength or is more quickly classified as having been seen before,
There should be a larger and earlier effect for massed than spaced repetitions relative to new items (or initial presentations) due to the higher perceived memory strength of massed items.  This difference in voltage and latency aligns with the idea that massed items are primed and in working memory whereas spaced items are not.  This effect would support deficient processing if it is an indicator not to process the information further.

% % dropping this to keep it simple
% On the other hand, study-phase retrieval requires episodic retrieval during a spaced repetition while deficient processing and contextual variability do not.  Because spaced repetitions need to be retrieved from long-term memory but massed do not, if the LPC indexes retrieval then study-phase retrieval theory might predict a larger effect for spaced repetitions than for massed.  Based on the prior stimulus repetition research showing a larger effect for massed items, we expect the deficient processing explanation to occur.

Theta power (4--7~Hz; frontal, temporal, and parietal) is related to memory formation and retrieval, particularly in medial temporal lobe regions (\citeNP{Klim1999,KlimEtal1996b,KlimEtal2006,LongEtal2014a,SedeEtal2003}; for a review, see \citeNP{MitcEtal2008a}).  It is also thought to signify item--context binding \cite{HansEtal2009a,HansEtal2011a,StauHans2013,SummMang2005}.  Theta power typically increases during encoding for subsequently remembered items (a positive SME), though negative SMEs have been demonstrated (e.g., \citeNP{BurkEtal2013,LegaEtal2011}; for a review, see \citeNP{HansStau2014}).

Theta power should increase for spaced compared to massed repetitions, but for different reasons under each hypotheses.  Under deficient processing, theta would decrease for massed repetitions simply because the item is not being retrieved or re-encoded well.
Under both contextual variability and study-phase retrieval, theta would increase for spaced repetitions because the intervening context (and the prior presentation in the latter case) is also encoded (new information, item--context binding).
However, there is a difference between these theories: given that a repetition is properly re-encoded, in comparison to the initially encoded memory, the ``contents'' of the two encoding events will be more similar under study-phase retrieval and more variable under contextual variability.  This brings us to the next topic.

% Another theta paper to read: \cite{HsieRang2014}


%%%%%%%%%%%%%%%%%%%%%%%%%%
% Similarity of study repetitions, test trials
%%%%%%%%%%%%%%%%%%%%%%%%%%

\section{Memory reinstatement and contextual variability assessed via neural similarity}

We do not just want to know about attention, the extent of semantic processing, and memory strength; we also want to investigate the contents of memory and whether it evolves across study repetitions.
We can address these areas by assessing whether the similarity of neural activity for an initial presentation and repetition is correlated with subsequent memory performance, and whether this similarity is modulated by spacing.  These assessments will bear on both study-phase retrieval and contextual variability, and can be tested both as an effect of similarity (``Does greater similarity or greater variability in neural activations during encoding lead to better subsequent memory?'') and as an interaction with spacing (``Do spaced and massed repetitions benefit from greater similarity or greater variability during encoding?  Are there differences?'').

% FUTURE:
% Although each theory predicts better subsequent memory performance (recognition, recall, etc.) for spaced compared to massed learning (except at very long retention intervals) and naturally includes encoding mechanisms as reasons for the effect, not all implicate processing at the time of test as having an impact on memory performance.  The similarity between an item's study repetitions and its subsequent memory test event is a meaningful comparison for contextual variability and study-phase retrieval.  For a spaced item, encoding variability would predict that the test is similar to only one of the study presentations, while it would likely be similar to both under study-phase retrieval.  Comparisons for massed items would be more difficult to interpret since context would not have drifted much.

% (e.g., an automatic \textit{vs} voluntary process, usually denoted by the effect occurring during incidental and intentional learning, respectively).

% Thus, the theories diverge in their predictions regarding the study and test conditions under which a spacing effect should occur.
% or the behavioral and physiological effects that should accompany spaced and massed learning.
% The theories can be supported or challenged by extending them to make predictions about the patterns of neural activity that should be seen during the study and test phases.  These predictions will be based on existing behavioral and neural data as well as formal models that account for the spacing effect, and will inform the proposed EEG analysis methods.

% \hl{If greater similarity between spaced repetitions compared to massed repetitions leads to better memory performance, SPR likely plays a role; if decreased similarity leads to better memory, CV likely plays a role.}

There are different ways to measure the similarity of neural activity between individual events.  For example, Representational Similarity Analysis \cite<RSA;>{KrieEtal2008} has been used in the fMRI literature recently \cite<e.g.,>{XueEtal2010} but has hardly been applied to EEG \cite<e.g.,>{GroeEtal2012,SuEtal2012} and has never been used in a memory experiment.  Other multivariate analyses have been used in electrocorticography (ECoG, or intracranial EEG) in relation to contextual drift \cite<e.g.,>{MannEtal2011}, and would be suitable here if used appropriately.  Regardless of the method used, a measure of neural similarity will help assess the nature of neural representations detectable at the scalp.  A challenge of this analysis is that it is difficult to know whether similarity measures are comparing item features, contextual features, the currently engaged mechanisms, or some combination of these.
% FUTURE:
% this analysis would benefit from separating item and context features in neural representations.

%%%%%%%%%%%%%%%%%%%%%%%%%%
% Classification
%%%%%%%%%%%%%%%%%%%%%%%%%%

% \section{Goal 4: Memory reinstatement assessed via pattern classification}

% Recorded patterns of neural activity have been shown to reflect the information that is currently being encoded or simply thought about as measured by both fMRI \cite{KuhlEtal2012b} and EEG \cite{KetzEtal2013,MortEtal2012}.  Thus, it may be possible to detect the reinstatement of a prior study episode during a repetition (as in study-phase retrieval).  For example, in a paired associates paradigm where the paired items are presented successively (A, then B), if a pair A--B is retrieved during the repetition A' (just prior to B'), perhaps it is possible to see activity related to B during A'.  This method potentially speaks to all three theories, but this could not be revealed using univariate statistics.
% Multivariate statistical methods like pattern classification can be used to detect and analyze distributed patterns of neural activity that represent, for example, stimulus categories (category B during A').


% Whether this reinstatement occurs differently during spaced and massed repetitions is unknown.  If pattern classifier accuracy was high during massed but not spaced repetitions, this might support contextual variability, implying that context has not drifted far for a massed repetition but it has for a spaced repetition.  Study-phase retrieval would be supported if reinstatement was stronger during a spaced repetition compared to a massed repetition.
% Importantly, this pattern would need to be differentiated from that expected under deficient processing where activity during a massed repetition is also decreased relative to spaced.


% Because the brain is a dynamical system that incorporates a constantly changing array of cognitive processes and distributed representations, running multiple neural activity classifiers on the same data could measure the engagement of these processes.  For example, there could be one classifier for attention, one for semantic processing, one for encoding, and one for retrieval; each would make assumptions about when effects occur in time and where they occur in space \cite{BorsEtal2013,KingDeha2014}.

% This goal has not yet been analyzed.

%%%%%%%%%%%%%%%%%%%%%%%%%%%%%%%%%%%%%%%%%%%%%%%%%%%%%%%%%%%%
% Final intro summary
%%%%%%%%%%%%%%%%%%%%%%%%%%%%%%%%%%%%%%%%%%%%%%%%%%%%%%%%%%%%

\section{Summary}

% \hl{Brief exposure to experiment design. Recap processes involved in each theory and how they will be assessed.}

The overarching goal of this thesis is to assess the major hypotheses for why the spacing effect occurs by examining data from experiments that capture this effect.
% as supported by behavioral effects and patterns of neural activity predicted under spaced and massed learning.
We recognize that we are relying on reverse inference to test these psychological theories in terms of of the spacing effect \cite{Pold2006,PoldWagn2004}: patterns of neural activity are used to make assumptions about active cognitive processes, as related to prior research.  Because there is so little research on the EEG correlates of the spacing effect, reverse inference provides an initial direction for our analyses.
To briefly recap the factors that likely influence each theory, deficient processing emphasizes attention and semantic processing, contextual variability emphasizes contextual drift,
% and study--test matching,
and study-phase retrieval emphasizes episodic retrieval of the prior presentation during a repetition.
We expect that both theory- and data-driven analyses of ERPs, oscillatory power, single trials, and
% study--study and study--test
neural similarity will support or challenge the theories, and will provide insight into the cognitive underpinnings of the spacing effect.

To briefly introduce the experiments presented here,
Experiment~1 involved a paired associates memorization task in which participants studied word--image pairs at two points in time in either a spaced or a massed fashion.  After a brief distractor task, a cued recall test was given where participants were required to remember the word originally paired with each image.
ERP and oscillatory effects were used to assess how attention, semantic processing, and memory retrieval and encoding mechanisms operate under spaced and massed learning and as modulated by subsequent memory.  Neural similarity was performed via dimensionality reduction techniques to attempt to understand memory content and contextual reinstatement.
Experiment~2 was an extension run using additional study repetition lags and used a similar design as well as similar analyses.

%Finally, the associative memory design will allow for the assessment of memory reinstatement during repetitions; this analysis has not yet been run.

% Two possibilities are proposed for Experiment~2.  Both predict the replication of effects from the first experiment where possible, as would be expected if we are analyzing the true correlates of spacing effect mechanisms.  The first replicates and extends Experiment~1 using additional inter-study lags.  Its purpose is to show that the behavioral and neural effects vary with the spacing effect as would be expected under each theory.  The second option takes an established spacing effect paradigm from the literature \cite[Experiment~1]{CepeEtal2009} and adds EEG recordings.  It is a multi-session experiment and has the added benefit of being more applicable to real world learning and simplifying the possibilities for why the spacing effect occurs (deficient processing should essentially be eliminated).  Options are proposed for adding a ``massed'' condition as either a between- or within-subjects manipulation.

