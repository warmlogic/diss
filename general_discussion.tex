% !TEX root = ./diss.tex

\chapter{General discussion}

\section{Summary of results}

\hl{(summarize the results in a way that integrates Exp 1 and 2 for each DV.  Then, for each DV you potentially have an interpretation with regard to whether or not the overall pattern supports/contradicts particular hypotheses.  As I say within, I might take the N1 results more seriously overall since they are significant in E1 but only marginal in E2.  Don’t go overboard, but this is certainly suggestive enough to not completely dismiss based on marginal E2 results alone.)}

\hl{(After you do the above for each DV, you should then be in the position to summarize patterns of support for the various hypotheses, and then you can revisit possible connections to particular theories. Its great that these theories suggest multiple mechanisms and your results also provide support for several of these mechanisms.  As with any general discussion, you will need to do a fair amount of rehashing main points from the introduction as you try to integrate your results with the literature etc.)}

%diverged slightly
\cbstart

The results of the two experiments followed the same overall pattern.  Based on the behavioral results of Experiment~2, we hoped that the additional spaced study lags would have provided more graded neural effects, but the analyses undertaken here do provide additional insights.  To reiterate at a high level, it seems like information that is no longer in working memory shows evidence of retrieval during a study repetition, and this benefits performance on a subsequent test.

Take marginal Experiment~2 N1 results relatively seriously, in combination with significant results of Experiment~1.

Plenty of previous research has shown theta to be important for memory processing.
\citeA{FuenEtal2010} showed that memory reactivation involves increased theta power, coupled with activity in the beta and gamma bands.
The tasks in the present experiments have not been working memory tasks requiring the maintenance and manipulation of information over short periods of time, but the reactivation of memory representations during stimulus repetitions correlates with lag (and with behavioral performance); longer lags showed more theta power.
Theta synchrony along the visual ventral stream (occipitotemporal region) is important for maintaining cortical representations \cite{DuzeEtal2010}, which is one reason why elevated levels of theta in the medial temporal lobes prior to stimulus presentation can lead to better memory encoding.





% Theta effects support SPR and CV, but not DP.  Voltage similarity supports SPR.  TF similarity could support CV.

% \cite{JafaEtal2013} MVPA theta review.

% The present set of experiments failed to fully disambiguate the roles of XYZ in the spacing effect.

% It would be enlightening to examine the effects that are related to the spacing effect.  Theta power seems to play a large role here.

% Perhaps there were no effects of contextual variability because of how the stimuli were encoded.  In a behavioral experiment where participants learned??? \citeA{HuffBodn2014} showed that when pairs of items are encoded while focusing on their shared characteristics (as might have occurred in the present experiments), there were no CV effects.

\section{Questions to address}

Does working memory (LPC) really apply to medium- and long-spaced repetitions?

Is it possible that the N400 and LPC are related? They show similar patterns of effects.

\section{Spacing effect practicalities}

In relation to deficient processing, it is important to keep in mind that since the processes involved in this hypothesis only seem to affect immediate repetitions.  Thus, deficient processing should provide almost no bearing on real-world spacing effects in which a learner typically has at least some separation between study episodes, regardless of how minor.
% where the information is surely no longer in working memory

\hl{(Education... Not sure whether it would be interesting to include text here. Maybe I've already written something...)}

\section{Future directions}


For future analyses, comparing the similarity of each during test to its study presentations would be another way to examine contextual reinstatement \cite{HowaKaha2002,LohnKaha2014b}.  This analysis would have a bearing on study-phase retrieval and contextual variability.
% ``encoding variability''  (p. 17) > ``contextual variability''
Contextual variability would be supported if a spaced test image is similar to only one study image, whereas study-phase retrieval would be supported if the test image was similar to both study images.
This analysis would benefit from separating item and context features in neural patterns.

% Another interesting analysis would be to run the current behavioral data in a model like MCM \cite{MozeEtal2009} and compare our relatively short learning curves to what would be optimal.

For future experiments, it would still be enlightening to get away from a paradigm that involves immediate repetition so that deficient processing could be eliminated as a possible theory.
An interesting way to take this would be to use an established spacing effect paradigm from the literature \cite<e.g.,>[Experiment~1]{CepeEtal2009} and add EEG recordings.  This would involve a multi-session experiment and have the added benefit of being more applicable to real-world learning.  It may also simplify the possibilities for why the spacing effect occurs (deficient processing should essentially be eliminated).
%A ``massed'' condition could be added as either a between- or within-subjects manipulation \hl{in this way (look at proposal)}.

% \hl{More interesting would be a paradigm with external validity, something that has more of a real-world learning timescale (see Experiment~2b), but this may be too far removed from the current paradigm to make accurate predictions.}

\cbend
