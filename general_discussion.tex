% !TEX root = ./diss.tex

\section{General Discussion}

The results of the two experiments presented here (were mostly the same) (diverged slightly).  Why?


Theta effects support SPR and CV, but not DP.  Voltage similarity supports SPR.  TF similarity could support CV.

\cite{JafaEtal2013} MVPA theta review.

\citeA{FuenEtal2010} showed that memory reactivation involves increased theta power, coupled with activity in the beta and gamma bands.
The tasks in the present experiments have not been working memory tasks requiring the maintenance and manipulation of information over short periods of time, but the reactivation of memory representations during stimulus repetitions correlates with lag (and with behavioral performance); longer lags show more theta power.

Theta synchrony along the visual ventral stream (occipitotemporal region) is important for maintaining cortical representations \cite{DuzeEtal2010}.  Elevated levels of theta in the medial temporal lobes prior to stimulus presentation can lead to better memory encoding.


The present set of experiments failed to fully disambiguate the roles of XYZ in the spacing effect.


It would be enlightening to examine the effects that are related to the spacing effect.  Theta power seems to play a large role here.


Perhaps there were no effects of contextual variability because of how the stimuli were encoded.  In a behavioral experiment where participants learned??? \citeA{HuffBodn2014} showed that when pairs of items are encoded while focusing on their shared characteristics (as might have occurred in the present experiments), there were no CV effects.

\subsection{Future Directions}


Also: similarity of test vs study (P1 and P2) - CV and SPR
``encoding variability''  (p. 17) > ``contextual variability''
CV: spaced test is similar to only one study item, SPR: similar to both.
This analysis would benefit from separating item and context features in neural patterns, but this is probably not possible (future direction).


It would still be enlightening to get away from a paradigm that involves immediate repetition so that deficient processing could be eliminated as a possible theory. 

% \hl{More interesting would be a paradigm with external validity, something that has more of a real-world learning timescale (see Experiment~2b), but this may be too far removed from the current paradigm to make accurate predictions.}

An interesting spacing effect experiment would be to take an established spacing effect paradigm from the literature \cite<e.g.,>[Experiment~1]{CepeEtal2009} and add EEG recordings.  This would involve a multi-session experiment and have the added benefit of being more applicable to real-world learning and simplifying the possibilities for why the spacing effect occurs (deficient processing should essentially be eliminated).  A ``massed'' condition could be added as either a between- or within-subjects manipulation \hl{in this way (look at proposal)}.

