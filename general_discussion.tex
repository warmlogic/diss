% !TEX root = ./diss.tex

\chapter{General discussion}

% \section{Questions to address}

% Does working memory (LPC) really apply to medium- and long-spaced repetitions?

% Is it possible that the N400 and LPC are related? They show similar patterns of effects.

% \hl{(TODO: See OlicEtal2000 page 3/1950 for disambiguation setup of N400 and LPC.)}

\section{Summary of results}

% \hl{(summarize the results in a way that integrates Exp 1 and 2 for each DV.  Then, for each DV you potentially have an interpretation with regard to whether or not the overall pattern supports/contradicts particular hypotheses.)}

The results of the two experiments are summarized below.  Overall, they followed the same pattern both behaviorally and physiologically, and there were no major discrepancies.  This was expected because the experiments and analyses were similar.  At a high level, it seems that memory performance on a subsequent test shows a benefit for additional exposure to information that is no longer in working memory (i.e., spaced repetitions) compared to massed repetitions due to the former being retrieve from long-term memory and receiving additional semantic processing.

Behaviorally, there were strong spacing (Experiment~1 and 2) and lag (Experiment~2) effects showing that long-term memory recall gains more of an advantage as lag increases.
Regarding the analysis of the additional spaced study lags in Experiment~2, we hoped to find more graded lag effects in the neural data.  Those we did find are synthesized below, and the analyses for this experiment do provide additional insights.

While the results across EEG analysis modalities (ERP, time--frequency, and similarity for each) are not directly comparable, we should not evaluate them in a vacuum.  We take a relatively conservative approach when integrating these results given the consistency of support for particular hypotheses; assessing every pattern across modalities would likely lead to too many interpretations of the data.
% has the potential to explode combinatorically.


\subsection{ERP results}

% Take marginal Experiment~2 N1 results relatively seriously, in combination with significant results of Experiment~1.

% The ERP effects in Experiment~1 and Experiment~2 aligned well overall.

\subsubsection{N1}

% N1 - Exp1 support DP. Exp2 supports DP (marginal effects, but the Exp1 pattern persisted). indexes early attentional mechanisms.

The visual N1 ERP component, which is related to early attentional processes, showed that massed repetitions attenuated compared to the larger negative deflection for spaced repetitions, specifically for short (2) and medium (12) lags.  While the effects were marginal in Experiment~2, the overall pattern from Experiment~1 persisted and should not be dismissed. (We note again that almost all ERP component voltages attenuated in the second experiment, perhaps because of slightly different electrode net placements across the two sessions.)  In terms of attentional processing, this indicates that spaced repetitions receive more attention than massed, which supports the deficient processing hypothesis: massed are not processed to the extent that spaced are because they feel familiar and their representations are still primed.  Also, spaced repetitions may only receive additional attention when they are recognized as being repeated, which may happen less frequently at longer lags.  It seems possible that the N1 leads to the subsequent engagement of semantic processes, especially when \citeA{ProvAdor2009} showed that a component with similar temporal and spatial characteristics has neural generators in regions that support semantic processing (BA10).

\subsubsection{N400}

% N400 - Exp1 supports DP. Exp2 DP provides only a partially explanation of the spacing effect because there was no lag effect. supports the activation of semantic information for spaced items, construction of a semantic representation.  massed items do not receive additional semantic processing events and may be at a disadvantage because of this.

The N400 ERP component, which is related to semantic processing, attenuates when the information being processed is in working memory, reflecting that the required semantic information is already activated.  In both experiments the N400 attenuated for massed compared to spaced repetitions, indicating a decrease in semantic processing.  This is in line with deficient processing. 
Additionally, spaced items showed SMEs at medium (12) lags in both experiments, so it seems that while massed items are put at a disadvantage overall, memory is also better when semantic processing engages to a greater extent.

\subsubsection{LPC}

% LPC - Exp1 indexes WM. Exp2 indexes WM. tells processing to disengage? does not support SPR, a possibility broached in the introduction.


The LPC, which is related to consciously recognizing information, showed effects that align with an explanation that centers around working memory.  Overall, massed repetitions showed larger peaks than spaced repetitions as well as larger repetition effects, meaning that massed repetitions were accessed to a greater extent compared to the other conditions.
% This component was not discussed from a working memory perspective in previous analyses in repetition experiments \cite[rather, they described it as ``template matching'']{OlicEtal2000,VanSEtal2007}, though our pattern of effects follows this idea.
% LAG EFFECTS
Experiment~2 also showed a voltage lag effect.  SMEs (recalled \textit{vs.} forgotten) decreased with lag, likely indicating that less information is recognized at longer lags, which is in line with spacing effect research involving longer lags \cite<e.g.,>{Glen1979}.
There was also a lag effect for LPC peak latency, where the peak occurred later for medium (12) and long (32) spaced compared to massed repetitions (massed did not differ significantly from short (2) spaced repetitions).  This shows a separation between shorter and longer lags.  It seems reasonable that primed items are recognized faster than those whose representations are no longer in working memory, and short (2) spaced repetitions may still be slightly primed.  To connect these effects to the hypotheses, it seems possible that a repetition's match to information in working memory may be an indicator for additional processing not to engage, which would support deficient processing.

\subsection{Time--frequency results}

\subsubsection{Theta}

Previous research has shown theta to be important for memory processing.
\citeA{FuenEtal2010} showed that memory reactivation involves increased theta power, coupled with activity in the beta and gamma bands.
The tasks in the present experiments have not been working memory tasks requiring the maintenance and manipulation of information over short periods of time, but longer lags showed more theta power and so the reactivation of memory representations during stimulus repetitions as indexed by theta seems to correlate with lag (and with behavioral performance).
Theta synchrony along the visual ventral stream (occipitotemporal region) is important for maintaining cortical representations \cite{DuzeEtal2010}, which is one reason why elevated levels of theta in the medial temporal lobes prior to stimulus presentation can lead to better memory encoding.

Effects in the theta band were consistent across experiments.  Both showed sustained theta synchrony across spaced word repetitions, which extended into image presentations, compared to massed repetitions.  There was particularly high theta power throughout long (32) spaced repetitions in Experiment~2, and theta power seems to decrease as lag decreases.
% In Experiment~1, spaced items (which had lag~$=12$) also showed this maintenance of theta power, so it is curious that medium (12) spaced items in Experiment~2 did not also show this sustained effect.
Because theta is involved in memory retrieval and encoding, and because these effects are seen when word--image associations are being formed (participants are asked to associate each word--image pair), this leads to the idea that study-phase retrieval and re-encoding occur more for spaced compared to massed repetitions.  This explanation makes sense because the ERP analyses showed that massed items are still in working memory, so they do not need to be retrieved.
These interesting theta effects may be an important part of why repetitions at longer lags are remembered better on average.

% LAG EFFECTS
% We did see an interesting lag effect in Experiment 2: for word presentations, pairwise comparisons in the three-way interaction [$n.s.$] showed more theta power for recalled long (32) spaced items compared to all other conditions, perhaps indicating why this condition is remembered best on average.

% Sustained theta synchrony at longer lags denotes that information is retrieved from long-term memory for spaced repetitions.

Regarding the negative theta SME for massed image repetitions, such an effect in this direction has been demonstrated under variable contexts \cite{StauHans2013}, but because these are immediate repetitions it seems unlikely that there has been much contextual drift across the two presentations.  If this effect is the result of a contextual difference, participants may have considered the stimuli differently on each presentation.  It is difficult to make strong claims about the negative SME from this perspective because we did not measure how participants perceived stimuli.

\subsubsection{Lower alpha}

In lower alpha, which generally correlates negatively with attentional processes (desynchronization indicates increased attention), massed word and image repetitions showed desynchronized more than spaced.  This goes against the prediction that deficient processing would make, that attention to massed items should decrease.  There was also a negative SME, where recalled items desynchronized more than forgotten ones, which is in line with lower alpha's connection to attention.

The only difference between the experiments in lower alpha was for image repetitions: Experiment~1 showed that massed images were more negative than spaced only in the first time window, while in Experiment~2 massed were lower overall.  This is only a quantitative difference, and the two experiments show similar patterns.  Interestingly, spaced images on average decreased more from the first to the second time window compared to massed images.  Perhaps the relative amount of desynchronization during stimulus processing is a factor in attentional effects.

\subsubsection{Upper alpha}

In the upper alpha band analyses of both experiments, massed word repetitions desynchronized more than spaced, though all conditions desynchronized, indicating that semantic processing is occurring.

There were some qualitative differences between the experiments in the upper alpha band during image repetitions.  Experiment~1 supported the idea that spaced items received more semantic processing (desynchronized more), but massed repetitions in Experiment~2 stayed desynchronized for longer.  The latter effect may have something to do with primed semantic representations in working memory, but there is no good reason it would not happen in both experiments.  Also in Experiment~2, there was no three-way interaction between spacing, time, and memory, only a two-way interaction between time and memory.  The important part of this interaction in Experiment~1 was that recalled spaced items in the middle time window desynchronized more than recalled massed items in all time windows.  Experiment~2 did show a larger SME in the latest time window (spacing was not a significant factor), which still support the idea that recalled items are processed more thoroughly at a semantic level (more desynchronization) than forgotten items.


% Also in the upper alpha band, while there was no main effect of spacing, Experiment~2 showed a memory effect.

\subsubsection{Lower beta}

Lower beta effects were mainly the same across the two experiments.  Subsequently remembered words and images desynchronized more than forgotten ones, implying---like upper alpha---that more semantic processing occurs.
Massed trials get more semantic processing during word repetitions (middle and late time windows) perhaps because these representations are primed, but all conditions desynchronize during the images, implying that massed and spaced are processed approximately equally.  Overall, upper alpha and lower beta effects demonstrate the importance of semantic processing to subsequent memory performance.

\subsection{Similarity results}

Similarity analyses involved comparisons of neural activity for individual trials during the initial and the repetition presentations of image stimuli.  This method attempted to investigate the active information at a relatively abstract level.

\subsubsection{Voltage}

Voltage similarity results were consistent across the experiments.  Spaced items (regardless of lag in Experiment~2) maintained similarity during the initial and repetition image presentations, whereas the similarity of massed items dropped off significantly.  These results support the idea that spaced repetitions are retrieved from memory and processed in a similar way, and this does not occur for massed items.  Thus, these results support study-phase retrieval of spaced items and deeply challenge the predictions of contextual variability if we expect that equivalent levels of representational similarity across spaced repetitions indicate consistent contextual encoding.

\subsubsection{Time--frequency}

For time--frequency data, the results showed a relatively complicated pattern of results, particularly in Experiment~2, and it is difficult to distill an overall picture.  It seems that longer lags may have lower similarity, which would align with contextual variability since context has drifted across time.
On the other hand, the first experiment showed a three-way interaction between spacing, time, and memory, which seems to support study-phase retrieval in that remembered items are more similar.  However, there was no such interaction in Experiment~2.  Overall, these pattern are not clear enough within or across experiments to make any hard conclusions.



%%%%%%%%%%%%%%%%%%%%%%%%%%%%%%%%%%%%%%%%%%%%%%%%%%%%%%%%%%%%%%%
% HYPOTHESES
%%%%%%%%%%%%%%%%%%%%%%%%%%%%%%%%%%%%%%%%%%%%%%%%%%%%%%%%%%%%%%%

\section{Alignment with the hypotheses}

% \hl{(After you do the above for each DV, you should then be in the position to summarize patterns of support for the various hypotheses, and then you can revisit possible connections to particular theories. Its great that these theories suggest multiple mechanisms and your results also provide support for several of these mechanisms.  As with any general discussion, you will need to do a fair amount of rehashing main points from the introduction as you try to integrate your results with the literature etc.)}

When aligning the results from the present experiments with the three evaluated hypotheses, it is important to keep in mind that subsequent memory performance is a critical factor in the spacing effect.  Not all significant effects included memory as a significant factor, but we think they may still have a place in the broader story since these processes (mainly attentional and semantic processes) are clearly involved in the brain's distributed organization and are important for understanding information.

\subsection{Deficient Processing}

Many of the results summarized above support the deficient processing hypothesis.  The N1 showed that early attentional processes orient more to spaced items than massed, and similarly the N400 demonstrated this pattern for semantic processing.  Additionally, the LPC showed that massed repetitions are recognized earlier than spaced, which perhaps leads to decreased processing for massed items and/or additional processing for spaced items.  It is interesting that all of these effects involve ERP components, though we have no conclusion to draw from this observation.

On the other hand, assuming deficient processing would predict decreased attention for massed compared to spaced items in all modalities, this hypothesis is challenged by the lower alpha band effects that showed the opposite pattern.  However, because these effects occur relatively late compared to earlier attentional effects, they may be related to a different aspect of stimulus processing.

Deficient processing theory's reliance on working memory, which involves the attenuation of attentional mechanisms and semantic processing, as well as support garnered through explanations involving neural repetition suppression in fMRI experiments \cite{CallSchw2010,XueEtal2011}, makes it only seem like a viable explanation at short timescales because it would likely fade after a brief period of time.
Overall, the pattern of results supporting deficient processing explain what makes massed items less memorable rather than what makes spaced items more memorable.  This results from deficient processing effects differentiating massed and spaced conditions, and not generally showing lag effects.
It seems that the characterization of deficient processing as an ``impostor'' spacing effect by \citeA{DelaEtal2010} is an accurate label.

% Thus, this theory accounts for the spacing effect with a relative difference between the two conditions, and

\subsection{Contextual Variability}

% Theta effects support SPR and CV, but not DP.  Voltage similarity supports SPR.  TF similarity could support CV.

% Perhaps there were no effects of contextual variability because of how the stimuli were encoded.  In a behavioral experiment where participants learned??? \citeA{HuffBodn2014} showed that when pairs of items are encoded while focusing on their shared characteristics (as might have occurred in the present experiments), there were no CV effects.

None of the effects support the contextual variability hypothesis as being involved in the spacing effect.  In fact, it is challenged by the neural similarity analyses using voltage measurements.  Thus, this hypothesis does not seem like a contender for supporting the spacing effect.
% which supports study-phase retrieval.

As a side note regarding the negative theta SME for massed images, negative SMEs in the theta band have been induced when context is purposely varied \cite{StauHans2013}.  An idea approached earlier is that remembered massed image repetitions in both experiments may have had a more varied context (and a negative SME) because they were thought about in a different way.   In line with this, the similarity results for ERPs show that massed items decrease in similarity across time.  It may be beneficial to either understand or control how participants think about stimuli during study episodes.

% The time--frequency similarity results do not support this because the massed image study presentations increase in similarity across time.

%though this is not specific to recalled items
% and this is also integrating across modalities which may not be a valid comparison.

\subsection{Study-Phase Retrieval}

% \cite{JafaEtal2013} MVPA theta review.

It has been both proposed and demonstrated from relatively early in the theories and analyses around the spacing effect that recognition during a repetition is important for subsequent memory performance.  \citeA{Madi1969} demonstrated that only items recognized as old on their second presentations contributed to the spacing effect in free recall, and \citeA{Glen1979} showed that items which were not recognized at the second presentation were recalled very poorly.
% Glen1979 p111

In the present experiments, study-phase retrieval received support through some relatively clear effects.  The theta band is integrally involved in memory-related processing (both retrieval and encoding), and we saw that spaced repetitions maintained theta synchrony across time.  Additionally, the results of our neural similarity approach comparing single-trial representations give further support to study-phase retrieval.  The voltage effects clearly indicate that the neural representations
% as described by each trial's feature vector
maintain a consistently strong similarity for spaced items while their similarity quickly dissipates for massed items.  Additionally, in the time--frequency domain, that remembered items are more similar than forgotten items supports the idea that the reinstatement of a prior representation is important for subsequent memory.
These effects are in line with memory reinstatement, which has been demonstrated in both experimental settings \cite{MannEtal2011} as well as in models that account for the spacing effect \cite{LohnKaha2014b}.

\subsection{Additional considerations}

As mentioned above, some effects do not directly speak to the hypotheses but do inform patterns related to subsequent memory performance.  We found that, not so surprisingly, both increased attention (lower alpha) and semantic processing (upper alpha, lower beta) are important factors for remembering information at a later point.  Even though these mechanisms may not directly impact the spacing effect, they still have a bearing on how information is processed.

It is important to remember that the massed condition in experiments like these is relatively contrived.
% surely no one would argue that it is externally valid. 
That is to say, immediate repetitions like these are less likely to occur under real-world learning situations, though obviously it is possible to study the same information twice in a row.  Because deficient processing effects seem to dissipate after a relatively short amount of time, when evaluating explanations for the spacing effect in future research, this hypothesis should be critically considered when spaced conditions involve longer lags.

% In relation to deficient processing, it is important to keep in mind that since the processes involved in this hypothesis only seem to affect immediate repetitions.  Thus, deficient processing should provide almost no bearing on real-world spacing effects in which a learner typically has at least some separation between study episodes, regardless of how minor.
% % where the information is surely no longer in working memory

A final consideration regarding any effect that relies on a subsequent test is that the relation between learning and retrieval conditions will potentially have a profound impact on memory performance \cite{TulvThom1973}.  For example, performance for spaced and massed conditions may be differentially affected by the length of the retention interval \cite<the delay before the test;>{CepeEtal2006,Glen1976,Glen1977,Glen1979} or whether memory is tested using an experimenter-supplied retrieval cue (e.g., recognition, associative retrieval) or not (e.g., free recall) \cite{Glen1979,Gree1989a}.  We attempted to control for many potential factors that could impact performance and therefore impact hypothesis interpretations.  While it is possible that the results could change under different conditions, we feel that our results align well with the literature on the spacing effect as a whole.

\section{Future directions}

For future analyses, comparing the similarity of each image during the test/recall phase to its study presentations would be another way to examine memory reinstatement \cite{HowaKaha2002,LohnKaha2014b}.  This analysis could be done in an equivalent way to the similarity analyses presented here, and would provide additional additional assessments of study-phase retrieval and contextual variability.
% ``encoding variability''  (p. 17) > ``contextual variability''
Contextual variability would be supported if a spaced test image is similar to only one study image, whereas study-phase retrieval would be supported if the test image was similar to some combination of the study images.
The similarity analyses in general might benefit from separating item and context features in neural patterns, perhaps in a method similar to that of \citeA{MannEtal2011}.

% Another interesting analysis would be to use the present behavioral data in a model such as MCM \cite{MozeEtal2009}, which makes predictions about optimal spacings.  Then we could compare our relatively short learning curves to what would be optimal.
% those that the model predicts would result from optimal spacing repetitions.

%For future experiments, it would be enlightening to get away from a paradigm that involves immediate repetitions so that deficient processing could likely be eliminated as a possible theory.

An interesting direction to take a future experiment would be to make EEG recordings in an established spacing effect paradigm from the literature \cite<e.g.,>[Experiment~1]{CepeEtal2009}.  Here, the lag condition could be two contiguous days, while the spaced repetition could occur multiple days later.  This would necessarily involve a multi-session experiment, and would have the added benefit of being more applicable to real-world learning situations.  It also seems unlikely that deficient processing would play a role; if it was eliminated, this might simplify the possibilities for why the spacing effect occurs.

%A ``massed'' condition could be added as either a between- or within-subjects manipulation \hl{in this way (look at proposal)}.

% \hl{More interesting would be a paradigm with external validity, something that has more of a real-world learning timescale (see Experiment~2b), but this may be too far removed from the current paradigm to make accurate predictions.}


% \hl{(Education... Not sure whether it would be interesting to include text here. Maybe I've already written something...)}


\section{Conclusions}

Through a variety of mechanisms including attention, semantic processing, memory retrieval, and memory encoding, our results lend support to two hypotheses: deficient processing and study-phase retrieval.  When studying information for a second time at a spaced interval, the retrieval of the initial study episode from long-term memory and the additional semantic processing received benefits performance on a subsequent test compared to studying massed repetitions.


%%%%%%%%%%%%%%%%%%%%%
% BONEYARD!
%%%%%%%%%%%%%%%%%%%%%

% It would be enlightening to examine the effects that are related to the spacing effect.  Theta power seems to play a large role here.

% The attenuation of the visual N1 ERP component for massed compared to spaced repetitions implies that massed items may be processed at a level that is disadvantageous.  This supports the deficient processing hypothesis.

