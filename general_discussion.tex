% !TEX root = ./diss.tex

\chapter{General discussion}

% The present set of experiments failed to fully disambiguate the roles of XYZ in the spacing effect.

\section{Summary of results}

% \hl{(summarize the results in a way that integrates Exp 1 and 2 for each DV.  Then, for each DV you potentially have an interpretation with regard to whether or not the overall pattern supports/contradicts particular hypotheses.)}

The results of the two experiments followed the same overall pattern.  Based on the behavioral results of Experiment~2, we hoped that the additional spaced study lags would have provided more graded neural effects, but the analyses undertaken here do provide additional insights.  To summarize at a high level, it seems like information that is no longer in working memory shows evidence of retrieval during a study repetition, and this benefits performance on a subsequent test.

\subsection{ERP results}

% Take marginal Experiment~2 N1 results relatively seriously, in combination with significant results of Experiment~1.

The ERP effects in Experiment~1 and Experiment~2 aligned well overall, and there were no significant discrepancies.  We expected more in the way of lag effects in Experiment~2, but there were none that followed the full gradient.  The closest to a lag effect here was for LPC latency, where the peak occurred later for medium (12) and long (32) spaced compared to massed repetitions; massed did not differ significantly from short (2) spaced repetitions, implying that the effect may be different across lags.

% N1 - Exp1 support DP. Exp2 supports DP (marginal effects, but the Exp1 pattern persisted). indexes early attentional mechanisms.

The visual N1 ERP component, which is related to early attentional processes, showed that massed repetitions attenuated compared to spaced repetitions.  While the effects were only marginal in Experiment~2, the overall pattern from the Experiment~1 persisted and should not be dismissed. (We note again that almost all ERP component voltages were attenuated in the second experiment, perhaps because of slightly different electrode net placements across the two sessions.)

% N400 - Exp1 supports DP. Exp2 DP provides only a partially explanation of the spacing effect because there was no lag effect. supports the activation of semantic information for spaced items, construction of a semantic representation.  massed items do not receive additional semantic processing events and may be at a disadvantage because of this.

The N400 ERP component, which is related to semantic processing, attenuates when the information being processed is in working memory, reflecting that the required semantic information is already activated.

% LPC - Exp1 indexes WM. Exp2 indexes WM. tells processing to disengage? does not support SPR, a possibility broached in the introduction.

The LPC, which is related to consciously accessing information, showed effects that align with an explanation that centers around working memory.  This component was not described from a working memory perspective in previous analyses in repetition experiments \cite[rather, they discussed it as ``template matching'']{OlicEtal2000,VanSEtal2007}, but the pattern of effects follows this idea.


\subsection{Time--frequency results}

Effects in the theta band were consistent across experiments.  Both showed sustained theta synchrony across spaced word repetitions and into image presentations compared to massed repetitions, and this was particularly true for long (32) spaced repetitions in Experiment~2.  Because theta is involved in memory retrieval and encoding and that these effects are seen when word--image associations are being formed (participants are asked to associate the word and image), this leads to the idea that study-phase retrieval and re-encoding are occurring more for spaced repetitions compared to massed repetitions.  This explanation makes sense because our previous ERP analyses have shown that massed items are still in working memory.  Sustained theta synchrony at longer lags denotes that information needs to be retrieved from long-term memory for these repetitions.

Regarding the negative theta SME for massed repetitions, because these are immediate repetitions it seems unlikely that there has been much contextual drift across the two presentations.  If this effect is the result of a contextual difference, participants may have considered the stimuli differently on each presentation.  We did not measure how participants perceived stimuli, and thus it is difficult to make strong claims about the negative SME.

\subsection{Similarity results}


\section{Alignment with the hypotheses}

% \hl{(After you do the above for each DV, you should then be in the position to summarize patterns of support for the various hypotheses, and then you can revisit possible connections to particular theories. Its great that these theories suggest multiple mechanisms and your results also provide support for several of these mechanisms.  As with any general discussion, you will need to do a fair amount of rehashing main points from the introduction as you try to integrate your results with the literature etc.)}

\subsection{Deficient Processing}

Under deficient processing, we would expect that


The attenuation of the visual N1 ERP component for massed compared to spaced repetitions implies that massed items may be processed at a level that is disadvantageous.  This supports the deficient processing hypothesis.



\subsection{Contextual Variability}

% Theta effects support SPR and CV, but not DP.  Voltage similarity supports SPR.  TF similarity could support CV.


% Perhaps there were no effects of contextual variability because of how the stimuli were encoded.  In a behavioral experiment where participants learned??? \citeA{HuffBodn2014} showed that when pairs of items are encoded while focusing on their shared characteristics (as might have occurred in the present experiments), there were no CV effects.

Regarding the negative theta SME for massed images, negative SMEs in the theta band have been induced when context is purposely varied \cite{StauHans2013}.  An idea approached earlier is that remembered massed image repetitions in both experiments may have had a more varied context (and a negative SME) because they were thought about in a different way.  The time--frequency similarity results do not support this because the massed image study presentations increase in similarity across time.  However, the similarity results for ERPs show that massed items decrease in similarity across time, though this is not specific to recalled items and this is also integrating across modalities which may not be a valid comparison.

\subsection{Study-Phase Retrieval}

% \cite{JafaEtal2013} MVPA theta review.

Previous research has shown theta to be important for memory processing.
\citeA{FuenEtal2010} showed that memory reactivation involves increased theta power, coupled with activity in the beta and gamma bands.
The tasks in the present experiments have not been working memory tasks requiring the maintenance and manipulation of information over short periods of time, but the reactivation of memory representations during stimulus repetitions correlates with lag (and with behavioral performance); longer lags showed more theta power.
Theta synchrony along the visual ventral stream (occipitotemporal region) is important for maintaining cortical representations \cite{DuzeEtal2010}, which is one reason why elevated levels of theta in the medial temporal lobes prior to stimulus presentation can lead to better memory encoding.


\section{Questions to address}

Does working memory (LPC) really apply to medium- and long-spaced repetitions?

Is it possible that the N400 and LPC are related? They show similar patterns of effects.

\section{Spacing effect practicalities}

In relation to deficient processing, it is important to keep in mind that since the processes involved in this hypothesis only seem to affect immediate repetitions.  Thus, deficient processing should provide almost no bearing on real-world spacing effects in which a learner typically has at least some separation between study episodes, regardless of how minor.
% where the information is surely no longer in working memory

\hl{(Education... Not sure whether it would be interesting to include text here. Maybe I've already written something...)}

\section{Future directions}


For future analyses, comparing the similarity of each during test to its study presentations would be another way to examine contextual reinstatement \cite{HowaKaha2002,LohnKaha2014b}.  This analysis would have a bearing on study-phase retrieval and contextual variability.
% ``encoding variability''  (p. 17) > ``contextual variability''
Contextual variability would be supported if a spaced test image is similar to only one study image, whereas study-phase retrieval would be supported if the test image was similar to both study images.
This analysis would benefit from separating item and context features in neural patterns.

% Another interesting analysis would be to run the current behavioral data in a model like MCM \cite{MozeEtal2009} and compare our relatively short learning curves to what would be optimal.

For future experiments, it would still be enlightening to get away from a paradigm that involves immediate repetition so that deficient processing could be eliminated as a possible theory.
An interesting way to take this would be to use an established spacing effect paradigm from the literature \cite<e.g.,>[Experiment~1]{CepeEtal2009} and add EEG recordings.  This would involve a multi-session experiment and have the added benefit of being more applicable to real-world learning.  It may also simplify the possibilities for why the spacing effect occurs (deficient processing should essentially be eliminated).
%A ``massed'' condition could be added as either a between- or within-subjects manipulation \hl{in this way (look at proposal)}.

% \hl{More interesting would be a paradigm with external validity, something that has more of a real-world learning timescale (see Experiment~2b), but this may be too far removed from the current paradigm to make accurate predictions.}


%%%%%%%%%%%%%%%%%%%%%
% BONEYARD!
%%%%%%%%%%%%%%%%%%%%%

% It would be enlightening to examine the effects that are related to the spacing effect.  Theta power seems to play a large role here.

