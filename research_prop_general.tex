% !TEX root = ./diss.tex

\section{General Research Design}
% section 4

\subsection{Method}

% (See Erika's proposal)

% Participants, payment, etc

Approximately 40 undergraduate students at the University of Colorado Boulder were recruited for each of two experiments.  In EEG sessions, participants were either given course credit \hl{(AH: How many?)} or paid \$15 per hour, and all experiments lasted approximately 2.5 hours.  All research was conducted under the guidelines for human studies research at the University of Colorado Boulder.  The method for both experiments is nearly identical (random stimulus assignment, hand counterbalancing, EEG recording and processing, analysis, etc.).

% \subsubsection{Participants}
% \subsubsection{Materials}
% \subsubsection{Design}
% \subsubsection{Procedure}

\subsection{EEG Recordings and Analysis}

EEG recordings provide a fine-grained time course of the electrical
activity of neurons, on the order of milliseconds, which is important
for determining when neural processes occur with respect to behavioral
responses.  Event-related potentials and average oscillatory power were used for analyses, as well as single-trial data.

% For analysis, EEG will be averaged across brief
% epochs that are time-locked to events such as stimulus presentations,
% and these event-related potentials (ERPs) are compared between
% conditions to show differences in voltage deflections (known as ERP
% components).  Particular ERP components, which dissociate in time
% (relative to stimulus presentation) and space (at particular
% electrodes), have come to be associated with cognitive processes.

% The analysis of neural oscillations has become an important tool for cognitive neuroscientists \cite<for reviews, see>{HansStau2014,NyhuCurr2010}.
% Decomposing EEG into time--frequency measures using methods such as wavelet transforms
% % or multitapers
% provides a more nuanced (and higher-dimensional) representation of
% neural activity, allowing for an investigation of the dynamics of neural networks in greater detail.  Oscillatory power is used to measure
% both local synchronous activity and long-range communication between
% brain regions, and different frequency bands have come to be
% associated with particular cognitive processes.

% This proposal also relies on assessing the neural similarity at the trial level.  These analyses will be more exploratory in nature, as they have not been reported in the literature before for scalp EEG and have never been used in relation to the spacing effect.
% % Preliminary results exclude pattern classification methods described earlier (Goal 4).

